\pagenumbering{gobble}

\begin{titlepage}
\newgeometry{hmargin=1cm,vmargin=2cm,centering}
\begin{center}
\vspace*{2cm}
{\LARGE Dissertation}\\

\vspace{8mm}
{\Huge \bf Reactive Synthesis:}

\vspace{3mm}
%{\bf \huge {$\frak b$ranching~logic~{\!\LARGE\&\!}~parameter\emph{i}sed~systems}}
%{\bf \huge {$\frak{branching}$~logic~{\!\LARGE\&\!}~parameter\emph{i}sed~systems}}
%{\bf \huge {branc$\frak h$ing~logic~{\LARGE\&}~parameter\emph{i}sed~systems}}
{\bf \huge {branching~logic~{\LARGE\&}~parameter\emph{i}zed~systems}}
\vspace{8mm}

{\LARGE Ayrat Khalimov}

\vspace{1cm}

{\large
\begin{tabular}{rl}
Advisor:    & Roderick Bloem \\&Graz University of Technology, Austria\\[0.5ex]
Reviewer: & Sven Schewe \\&University of Liverpool, UK\\
Dean of Studies: & Denis Helic\\&Graz University of Technology, Austria\\
\end{tabular}}

\vfill

\begin{figure}[!ht]
\centerline{\includegraphics[width=3.3cm,keepaspectratio=true]{figures/title/tug}}
\end{figure}

{\large
Institute for Applied Information Processing and Communications\\
Graz University of Technology\\
A-8010 Graz, Austria\\
}

{\large Januar 2018}
\end{center}

%\noindent
%\underline{\hspace*{3cm}}\\
%{\footnotesize \copyright ~ Copyright 2017 by the author}

\end{titlepage}

\restoregeometry

\cleardoublepage

\pagestyle{plain}
\pagenumbering{roman}

%\vspace*{0.2cm}

\begin{center}
{\Large\bfseries Acknowledgements}
\end{center}
This work would not be possible without RiSE network (established by Roderick Bloem and \boxed{\text{Helmut Veith}}).
I stumbled upon the poster with the PhD position by chance,
during a relaxed walk at EPFL where I was doing an internship.\\
I am grateful to my advisor Roderick Bloem,
who honestly answered my questions, patiently directed me by asking questions and pitching ideas,
and who always listened.
Sasha Rubin showed how to be rigid and develop theories,
with Swen Jacobs we wondered a lot around parameterised synthesis bouncing the token from each other,
and Sven Schewe convinced me that tree automata are easy-peasy.\\
My colleagues Robert K\"onighofer, Georg Hofferek, and Bettina K\"onighofer helped
in the initial integration and changed my attitude towards people outside of Russia.
Our secretaries Martina Piewald, Melanie Blauensteiner, Ursula Urwanisch, and Angelika Wagner
enabled me to focus on my work without administrative distractions.\\
Dedicated to my grandfather Rauf and our large family.

\cleardoublepage

%\vspace{5cm}
\begin{center}
{\Large\bfseries Statutory Declaration}
\end{center}
\vspace{5mm}
\noindent
I declare that I have authored this thesis independently, that I have
not used other than the declared sources/resources, and that I have
explicitly marked all material which has been quoted either literally
or by content from the used sources.

\vspace{2cm}

\noindent
\begin{tabular}{ccc}
\hspace*{5cm}     & \hspace*{1.5cm}   & \hspace*{5.7cm}\\
\dotfill          &                 & \dotfill\\
place, date       &                 & signature\\
\end{tabular}



% --- English Abstract --------------------------------------------------


\cleardoublepage


%\begin{changemargin}{1.4cm}{1.4cm}

\begin{center}
{\Large\bfseries Abstract}
\end{center}
%\vspace*{1mm}
Reactive synthesis is an automatic way
to translate a human intention expressed in some logic
into a system of some kind.
This thesis has two parts, devoted to logic and to systems.

\parbf{Part I}
In 1963 Alonzo Church introduced the synthesis problem~\cite{Church63}
for specifications in monadic second-order logic.
%Later Amir Pnueli introduced linear temporal logic (LTL)~\cite{pnueli1977temporal}.
Nowadays most model checkers and synthesizers use linear temporal logic (LTL)~\cite{pnueli1977temporal}.
LTL reasons about system runs in a linear fashion.
With LTL we can ask ``does a run reach a particular state?'' or
``does a run visits a particular state infinitely often?''.
LTL is linear in its nature, leaving the designer without structural properties,
which are expressible in computation tree logic (CTL) and its generalization \CTLstar~\cite{ctl-origin,ctlstar-origin}.
%\CTLstar and CTL were introduced by Emerson and Halpern and Clarke~\cite{ctl-origin,ctlstar-origin}.
With \CTLstar we can ask
``does a run never visit a particular state but it has a \emph{possibility}
  to reach it?''.
Such properties are important---they allow for fine-tuning the system structure.

In Part I, we develop two new approaches to \CTLstar synthesis.
The first approach is an extension (actually, two) of
the SMT-based bounded synthesis~\cite{BS}.
We describe two extensions:
one follows bottom-up \CTLstar model checking,
another one follows the automata framework~\cite{ATA}.
Then we develop the approach that reduces \CTLstar synthesis to LTL synthesis.
The reduction turns any LTL synthesiser into a \CTLstar synthesiser.
The approaches were implemented and are available online.

\parbf{Part II}
Modern systems become more and more distributed.
Such distributed systems are typically parameterized by the number of processes:
they should work for any number of processes.
The parameterized synthesis problem~\cite{JB14} asks,
given a parameterized specification, to find a process template,
that can be cloned to form a correctly behaving system of any size.
At the core of the method is the cutoff reduction technique:
reduce reasoning about systems with an arbitrary number of processes
to reasoning about systems of a fixed cutoff size.
The intrinsic parameter, hidden in the parameterized synthesis problem,
is how the processes are connected and how they communicate,
i.e., the system architecture.

In Part II, we study parameterized synthesis for two system architectures.
The first architecture is guarded systems~\cite{EmersonK03}
and is inspired by cache coherence protocols.
In guarded systems,
processes transitions are enabled or disabled depending on the existence of other processes in certain local states.
The existing cutoff results~\cite{EmersonK03} for guarded protocols
are restricted to closed systems, and are of limited use for liveness properties.
We close these gaps and prove tight cutoffs for open systems
with liveness properties, and also cutoffs for detecting deadlocks.

The second architecture is token-ring systems~\cite{Emerso03},
where the single token circulates processes arranged in a ring.
The experiments with the existing parameterized synthesis method~\cite{JB14}
showed that it does not scale to large specifications.
First, we optimize the method by refining the cutoff reduction,
using modularity and abstraction.
The evaluation show several orders of magnitude speed-ups.
Second, we perform parameterized synthesis case study on the industrial arbiter protocol AMBA~\cite{AMBAspec}.
We describe new tricks%
---a new cutoff extension and decompositional synthesis---%
that, together with the previously described optimizations, allowed us to synthesize AMBA
in a parameterized setting, for the first time.

