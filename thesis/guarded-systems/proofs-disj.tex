\section{Proof Techniques for Disjunctive Systems} \label{gua:sec:proofs-disj}

\subsection{\LTLmX\ Properties without Fairness: Existing Constructions}
\label{gua:sec:ideas-disj-nofair}

We revisit the main techniques
of the original proof of Theorem~\ref{thm:disj-cutoff-pairs}~\cite{Emerson00}. 

\begin{lemma}[Monotonicity: Disj, \LTLmX, Unfair] \label{disj:le:NonFairDisjunctiveMono}
    For disjunctive systems:
    \begin{align*}
    &\forall n \geq 1:\\
    &(A,B)^{(1,n)} \models \pexists h(A,B_1)
    \ \Impl \
    (A,B)^{(1,n+1)} \models \pexists h(A,B_1).
    \end{align*}
\end{lemma}
\begin{proof}
Given a run $x$ of $(A,B)^{(1,n)}$,
we construct a run $y$ of $(A,B)^{(1,n+1)}$: 
copy $x$ into $y$ and keep the additional process in the initial state.
\end{proof}


As for the bounding lemma, we construct an infinite run $y$ of $\cutoffsys$ 
with $y \models h(A,B^{(1)})$, 
based on an infinite run $x$ of $\largesys$ with $n>c$ and $x \models h(A,B^{(1)})$. 
The idea is to copy local runs $x(A)$ and $x(B_1)$ into $y$, 
and construct runs of other processes in a way 
that enables all transitions along $x(A)$ and $x(B_1)$. 
The latter is achieved with the flooding construction.

\myparagraph{Flooding construction \cite{Emerson00}}
Given a run $x = (s_1,e_1,p_1), (s_2,e_2,p_2) \ldots$ of $\largesys$, let
$\visited_\mB(x)$ be the set of all local states visited by $B$-processes in $x$,
i.e., $\visited_\mB(x) = \{ q \in Q_B \| \exists m \exists i.\ s_m(B_i) = q \}$. 

For every $q \in \visited_\mB(x)$ there is a local run of \largesys, say $x(B_i)$,
that visits $q$ first, say at moment $m_q$. Then, saying that process 
$B_{i_q}$ of \cutoffsys \emph{floods $q$} means:
$$y(B_{i_q}) = x(B_i)\slice{1}{m_q}(q)^\omega.$$ 
In words: the run $y(B_{i_q})$ is the same as $x(B_i)$ until moment $\time_q$,
and after that the process never moves.

The construction achieves the following. 
If we copy local runs of $A$ and $B_1$ from $x$ to $y$, 
and in $y$ for every $q \in \visited_\mB(x)$ introduce one process that floods $q$, 
then: 
if in $x$ at some moment $\time$ there is a process in state $q'$, 
then in $y$ at moment $\time$ there will also be a process (different from 
$A$ and $B_1$) in state $q'$. Thus, every transition of $A$
 and $B_1$, which is enabled at moment $\time$ in $x$, will also be enabled in $y$. 

\begin{lemma}[Bounding: Disj, \LTLmX, Unfair] \label{disj:le:NonFairDisjunctiveBounding}
    For disjunctive systems:
    \begin{align*}
    \forall n \geq |B|+2:\ 
    (A,B)^{(1,|B|+2)} \models \pexists h(A,B_1)
    \ \ \Implied \ \ 
    \largesys \models \pexists h(A,B_1).
    \end{align*}
\end{lemma}

The proof of the lemma is from \cite[Lemma 4.1.2]{Emerson00}. 
We recapitulate it to introduce the notions of ``a process floods a state'', 
\destutter, \interleave, and ``process mimics another process'',
which are used in our proofs later.

\begin{proof}[Proof idea]
The lemma is proved by copying local runs $x(A)$ and $x(B_1)$,
and flooding all states in $\visited_\mB(x)$.
To ensure that at least one process moves infinitely often
in $y$, we copy one additional (infinite) local run from $x$. Finally, it may
happen that the resulting collection of local runs violates the interleaving 
semantics requirement. To resolve this, we add stuttering steps into local 
runs whenever two or more processes move at the same time, and we 
remove global stuttering steps in $y$. Since the only difference between 
$x(A,B_1)$ and $y(A,B_1)$ are stuttering steps, $y$ and $x$ satisfy the same $
\LTLmX$-properties $h(A,B^{(1)})$. 
Since $\card{\visited_\mB(x)} \leq 
\card{B}$, we need at most $1+\card{B}+1$ copies of $B$ in \cutoffsys.
\end{proof}

\begin{proof}
Let $c = |B|+2$ and $n \geq c$. Let $x=(s_1,e_1,p_1), (s_2,e_2,p_2) \ldots$ be a run of $\largesys$ that satisfies $\pexists h(A,B_1)$. We construct a run $y$ of the cutoff system $\cutoffsys$ with $y(A, B_1) \simeq x(A, B_1)$.

Let $\visited(x)$ be the set of all visited states by B-processes in run $x$: $\visited(x) = \{ q \| \exists m \exists i: s_m(B_i) = q \}$. 

Construct the run $y$ of \cutoffsys as follows.
\li
  \-[a.] We copy runs of $A$ and $B_1$ from $x$ to $y$:
         $y(A)=x(A)$, $y(B_1)=x(B_1)$;

  \-[b.] Since $x$ is infinite, it has at least one infinitely moving process, denoted $B_\infty$. Devote one unique process $B_\infty$ in \cutoffsys that copies the behaviour of $B_\infty$ of \largesys: $y(B_\infty)=x(B_\infty)$.

  \-[c.] For every $q \in \visited$, there is a process of \largesys, denoted $B_i$, that visits $q$ first, at moment denoted $m_q$. Then devote one unique process in \cutoffsys, denoted $B_{i_q}$, that \emph{floods $q$}: set $y(B_{i_q}) = x(B_i)\slice{1}{m_q}(q)^\omega$. In words: the run $y(B_{i_q})$ repeats exactly that of $x(B_i)$ till moment $m_q$, after which the process is never scheduled.

  \-[d.] Let any other process $B_i$ of \cutoffsys not used in the previous steps (if any) \emph{mimic} the behavior of $B_1$ of \cutoffsys: $y(B_i) = y(B_1)$.
\il
The figure illustrates the construction.\ak{\init should be flooded}
On the left is $\largesys$ and on the right is $(A,B)^{(1,|B|+2)}$
(i.e., $(A,B)^{(1,5)}$, since $|B|=3$ in the figure).

\begin{figure}[hp]
\centering
\scalebox{0.7}{
%\usetikzlibrary{calc,arrows,shapes,fit}

\begin{tikzpicture}
\tikzstyle{every label} = [font=\normalsize];
\tikzstyle{state} = [circle, inner sep=2pt, minimum size=2mm, draw=black, node distance=0.5cm]
\tikzstyle{hiddenstate} = [state, circle,inner sep=0pt,outer sep=0pt, minimum size=0pt, draw=gray]
\tikzstyle{topstate} = [hiddenstate, node distance=0.75cm]
\tikzstyle{state21} = [state, rectangle]
\tikzstyle{state22} = [state, diamond]
\tikzstyle{state23} = [state, regular polygon,circle]

%----------------------------------------------------------------------------------------------------------------------------


%T1_1
\node (t11_top) [topstate, label={[name=t11_top-label] $A_1$}] {};
% \draw ($(t11_top)-(.2,0)$) -- ($(t11_top)+(.2,0)$);
\node (t11_1) [hiddenstate, below of=t11_top] {};
\node (t11_2) [hiddenstate, below of=t11_1] {};
\node (t11_3) [hiddenstate, below of=t11_2] {};
\node (t11_4) [hiddenstate, below of=t11_3] {};
\node (t11_5) [hiddenstate, below of=t11_4] {};
\node (t11_6) [hiddenstate, below of=t11_5] {};
\draw[|-] (t11_top) -- (t11_6);

%T2_1
\node (t21_top) [topstate, right of=t11_top, label={[name=t21_top-label] $B_1$}] {};
% \draw ($(t21_top)-(.2,0)$) -- ($(t21_top)+(.2,0)$);
\node (t21_1) [hiddenstate, below of=t21_top] {};
\node (t21_2) [hiddenstate, below of=t21_1] {};
\node (t21_3) [hiddenstate, below of=t21_2] {};
\node (t21_4) [hiddenstate, below of=t21_3] {};
\node (t21_5) [hiddenstate, below of=t21_4] {};
\node (t21_6) [hiddenstate, below of=t21_5] {};
\draw[|-] (t21_top) -- (t21_6);


%T2_2
\node (t22_top) [topstate, right of=t21_top, label={[name=t22_top-label] $B_2$}] {};
\node (t22_1) [state21, below of=t22_top] {};
\node (t22_2) [hiddenstate, below of=t22_1] {};
\node (t22_3) [state22, below of=t22_2] {};
\node (t22_4) [hiddenstate, below of=t22_3] {};
\node (t22_5) [state23, below of=t22_4] {};
\node (t22_6) [hiddenstate, below of=t22_5] {};
\draw[|-] (t22_top) -- (t22_1);
\draw[] (t22_1) -- (t22_3);
\draw[] (t22_3) -- (t22_5);
\draw[] (t22_5) -- (t22_6);

% other processes placeholder
\node (t2i_top) [topstate, right of=t22_top] {};

% T_2_m
\node (t2m_top) [topstate, right of=t2i_top, label={[name=t2m_top-label] $B_{n-1}$}] {};
\node (t2m_1) [hiddenstate, below of=t2m_top] {};
\node (t2m_2) [state23, below of=t2m_1] {};
\node (t2m_3) [hiddenstate, below of=t2m_2] {};
\node (t2m_4) [state22, below of=t2m_3] {};
\node (t2m_5) [hiddenstate, below of=t2m_4] {};
\node (t2m_6) [hiddenstate, below of=t2m_5] {};
\draw[|-] (t2m_top) -- (t2m_2);
\draw[] (t2m_2) -- (t2m_4);
\draw[] (t2m_4) -- (t2m_6);

% T_2_n
\node (t2n_top) [topstate, right of=t2m_top, label={[name=t2n_top-label] $B_n$}] {};
\node (t2n_1) [hiddenstate, below of=t2n_top] {};
\node (t2n_2) [hiddenstate, below of=t2n_1] {};
\node (t2n_3) [hiddenstate, below of=t2n_2] {};
\node (t2n_4) [hiddenstate, below of=t2n_3] {};
\node (t2n_5) [hiddenstate, below of=t2n_4] {};
\node (t2n_6) [hiddenstate, below of=t2n_5, label={[name=t2n_6-label]below:$\infty$}] {};
\draw[|-] (t2n_top) -- (t2n_6);

%\draw[dashed] (t2n_2) -- (t2n_4);
%\draw[dashed] (t2n_4) -- (t2n_6);

\coordinate (t11_old-top) at (t11_top);
\coordinate (t11_old-bottom) at (t11_6);
\coordinate (t21_old-top) at (t21_top);
\coordinate (t21_old-bottom) at (t21_6);
\coordinate(t22_old-bottom) at (t22_6);
\coordinate (t2n_old-top) at (t2n_top);
%\coordinate(t2n_old-bottom) at (t2n_6);
\coordinate(t2m_old-bottom) at (t2m_6);
\coordinate(t2n_old-bottom) at ($(t2n_6-label)-(0,0.1)$);


% other processes line
\coordinate (t22_center) at ($(t22_3.center)$);
\coordinate (t2m_center) at ($(t2m_3.center)$);
\node (othrerssss) at ($(t22_center)+(0.8,0)$) {\textbf{\ldots}};
% \draw[dotted,semithick] ($(t22_center)+(0.4,0)$) to ($(t2m_center)-(0.3,0)$); % a bit cheating

% box
\node (box_outer) [draw=black, fit=(t11_top-label.center) (t2n_6), inner sep=0.5cm] {} ;
%\draw[solid] ($(t11_top-label)!0.5!(t21_top-label)+(0,0.5)$) to ($(t11_6)!0.5!(t21_6)-(0,0.5)$) ;
% \draw[solid] ($(t21_top-label)!0.5!(t22_top-label)+(0,0.5)$) to ($(t21_6)!0.5!(t22_6)-(0,0.5)$) ;



%----------------------------------------------------------------------------------------------------------------------------

%T1_1
\node (t11_top) [topstate, label={[name=t11_top-label] $A_1$}, right of=t2n_old-top, node distance=1.5cm] {};
% \draw ($(t11_top)-(.2,0)$) -- ($(t11_top)+(.2,0)$);
\node (t11_1) [hiddenstate, below of=t11_top] {};
\node (t11_2) [hiddenstate, below of=t11_1] {};
\node (t11_3) [hiddenstate, below of=t11_2] {};
\node (t11_4) [hiddenstate, below of=t11_3] {};
\node (t11_5) [hiddenstate, below of=t11_4] {};
\node (t11_6) [hiddenstate, below of=t11_5] {};
\draw[|-] (t11_top) -- (t11_6);

%T2_1
\node (t21_top) [topstate, right of=t11_top, label={[name=t21_top-label] $B_1$}] {};
% \draw ($(t21_top)-(.2,0)$) -- ($(t21_top)+(.2,0)$);
\node (t21_1) [hiddenstate, below of=t21_top] {};
\node (t21_2) [hiddenstate, below of=t21_1] {};
\node (t21_3) [hiddenstate, below of=t21_2] {};
\node (t21_4) [hiddenstate, below of=t21_3] {};
\node (t21_5) [hiddenstate, below of=t21_4] {};
\node (t21_6) [hiddenstate, below of=t21_5] {};
\draw[|-] (t21_top) -- (t21_6);


%T2_2
\node (t22_top) [topstate, right of=t21_top, label={[name=t22_top-label] $B_2$}] {};
\node (t22_1) [state21, below of=t22_top] {};
\node (t22_2) [state21, below of=t22_1] {};
\node (t22_3) [state21, below of=t22_2] {};
\node (t22_4) [state21, below of=t22_3] {};
\node (t22_5) [state21, below of=t22_4] {};
\node (t22_6) [hiddenstate, below of=t22_5] {};
\draw[|-] (t22_top) -- (t22_1);
\draw[dashed] (t22_1) -- (t22_2);
\draw[dashed] (t22_2) -- (t22_3);
\draw[dashed] (t22_3) -- (t22_4);
\draw[dashed] (t22_4) -- (t22_5);
\draw[dashed] (t22_5) -- (t22_6);

% other processes placeholder
\node (t2i_top) [topstate, right of=t22_top] {};

% T_2_3
\node (t23_top) [topstate, right of=t22_top, label={[name=t23_top-label] $B_3$}] {};
\node (t23_1) [state21, below of=t23_top] {};
\node (t23_2) [hiddenstate, below of=t23_1] {};
\node (t23_3) [state22, below of=t23_2] {};
\node (t23_4) [state22, below of=t23_3] {};
\node (t23_5) [state22, below of=t23_4] {};
\node (t23_6) [hiddenstate, below of=t23_5] {};
\draw[|-] (t23_top) -- (t23_1);
\draw[] (t23_1) -- (t23_3);
\draw[dashed] (t23_3) -- (t23_4);
\draw[dashed] (t23_4) -- (t23_5);
\draw[dashed] (t23_5) -- (t23_6);


% T_2_4
\node (t24_top) [topstate, right of=t23_top, label={[name=t24_top-label] $B_4$}] {};
\node (t24_1) [hiddenstate, below of=t24_top] {};
\node (t24_2) [state23, below of=t24_1] {};
\node (t24_3) [state23, below of=t24_2] {};
\node (t24_4) [state23, below of=t24_3] {};
\node (t24_5) [state23, below of=t24_4] {};
\node (t24_6) [hiddenstate, below of=t24_5] {};
\draw[|-] (t24_top) -- (t24_2);
\draw[dashed] (t24_2) -- (t24_3);
\draw[dashed] (t24_3) -- (t24_4);
\draw[dashed] (t24_4) -- (t24_5);
\draw[dashed] (t24_5) -- (t24_6);

% T_2_5
\node (t25_top) [topstate, right of=t24_top, label={[name=t25_top-label] $B_5$}] {};
\node (t25_1) [hiddenstate, below of=t25_top] {};
\node (t25_2) [hiddenstate, below of=t25_1] {};
\node (t25_3) [hiddenstate, below of=t25_2] {};
\node (t25_4) [hiddenstate, below of=t25_3] {};
\node (t25_5) [hiddenstate, below of=t25_4] {};
\node (t25_6) [hiddenstate, below of=t25_5, label={[name=t25_6-label]below:$\infty$}] {};
\draw[|-] (t25_top) -- (t25_6);


\coordinate (t11_new-top) at (t11_top);
\coordinate (t11_new-bottom) at (t11_6);
\coordinate (t21_new-top) at (t21_top);
\coordinate (t21_new-bottom) at (t21_6);
\coordinate (t22_new-top) at (t22_top);
\coordinate (t22_new-bottom) at (t22_6);
\coordinate(t23_new-bottom) at (t23_6);
\coordinate(t24_new-bottom) at (t24_6);
%\coordinate(t25_new-bottom) at (t25_6);
\coordinate(t25_new-bottom) at ($(t25_6-label)-(0,0.1)$);

% box
\node (box_outer) [draw=black, fit=(t11_top-label.center) (t25_6), inner sep=0.5cm] {} ;
%\draw[solid] ($(t11_top-label)!0.5!(t21_top-label)+(0,0.5)$) to ($(t11_6)!0.5!(t21_6)-(0,0.5)$) ;
% \draw[solid] ($(t21_top-label)!0.5!(t22_top-label)+(0,0.5)$) to ($(t21_6)!0.5!(t22_6)-(0,0.5)$) ;

\draw[->, bend right, dotted,semithick] (t22_old-bottom) to (t22_new-bottom);
\draw[->, bend right, dotted,semithick] (t22_old-bottom) to (t23_new-bottom);
\draw[->, bend right, dotted,semithick] (t2m_old-bottom) to (t24_new-bottom);
%\draw[->, bend right, dotted,semithick] (t2n_old-bottom) to (t25_new-bottom);
%\draw[->, bend left, dotted,semithick] (t2m_top-label.north) to (t24_top-label.north);
\draw[->, bend left=5, dotted,semithick] (t2n_top-label.north) to (t25_top-label.north);


\end{tikzpicture}

}
\end{figure}

The correctness follows from the observation that any transition of any process at any moment $m$ of $y$ was done by some process in $x$ at moment $m$, and hence is enabled at $m$. Also note that, if $\geq 2$ processes transit simultaneously in $y$, then the guards of their transitions will be enabled even if both of them are removed from the state space\ak{vague}. Note that it is possible that in $y$:
\li
  \- more than one process transits at the same moment. Then, \emph{\interleave} the transitions of such processes, namely arbitrarily sequentialize them. \ak{why are enabled}
  \- at some moment no processes move. Then remove elements of the run $y$ -- the resulting run is denoted $\destutter(y)$.
\il
This construction uses $|\visited| + 2 \leq |B|+2$ copies of B (ignoring case (d)).
\end{proof} 

\begin{tightness}[Disj, \LTLmX, Unfair] \label{obs:disj:tight_prop}
    The cutoff in Lemma~\ref{disj:le:NonFairDisjunctiveBounding} is tight.
    I.e., for any $k$ there exist process templates $(A,B)$ with $|B| = k$ 
    and $\LTLmX$ formula $h(A,B_1)$ such that:
    $$
    (A,B)^{(1,|B|+2)} \models \pexists h(A,B_1) ~~and~~ 
    (A,B)^{(1,|B|+1)} \not\models \pexists h(A,B_1).
    $$
\end{tightness}
\begin{proof}
The idea of the proof relies on the subtleties of the definition of a run: it is infinite (thus not globally deadlocked), and in each step of a run exactly one process moves. 

Consider the templates from Figure~\ref{fig:obs:disj:tight_prop} and let $\pexists h(A,B_1) = \pexists (\eventually 3_{B_1} \land \eventually\always (2_{B_1} \land {end}_A))$. In words: there exists a run in a system where process $B_1$ visits $3_B$ and process $B_1$ with $A$ eventually always stay in $2_B$ and ${end}_A$.
\begin{figure}[tb]
\centering
\begin{subfigure}[b]{0.35\textwidth}\center
\scalebox{0.66}{%!TEX root = table.tex
\begin{tikzpicture}[node distance=2.1cm,inner sep=1pt,minimum size=0.5mm,->,>=latex]


\node[initial below, state] (a_1) {$1_A$};
\node (dots) [right of=a_1] {$\ldots$};
\node[state] (a_all) [right of=dots] {${all}_A$};
\node[state] (a_end) [right of=a_all] {${end}_A$};


\path (a_1) edge [above] node {$\disj{1_B}$} (dots);
\path (dots) edge [above] node {$\disj{{\card{B}}_B}$} (a_all);
\path (a_all) edge [above] node {$\disj{3_B}$} (a_end);



\end{tikzpicture}
}
\caption*{Template A}
\end{subfigure}
\begin{subfigure}[b]{0.62\textwidth}\center
\scalebox{0.66}{%!TEX root = table.tex
\begin{tikzpicture}[node distance=2.1cm,inner sep=1pt,minimum size=0.5mm,->,>=latex]

\node[initial below, state] (b_1) {$1_B$};
\node[state] (b_2) [left of=b_1] {$2_B$};
\node[state] (b_3) [right of=b_1]{$3_B$};
\node (dots) [right of=b_3] {$\ldots$};
\node[state] (b_k) [right=2.3cm of dots] {${\card{B}}_B$}; 

\path (b_1) edge [above] node {$\disj{1_B}$} (b_2);
\path (b_1) edge [above] node {$\disj{1_B}$} (b_3);
\path (b_3) edge [above] node {$\disj{3_B}$} (dots);
\path (dots) edge [above] node {$\disj{{\card{B}{-}1}_B}$} (b_k);
\path (b_k) [loop above] edge [right] node {$\disj{{\card{B}}_B}$} (b_k);
\path (b_3) [bend right=60] edge [above] node {$\disj{{all}_A}$} (b_1);

\end{tikzpicture}
}
\caption*{Template B}
\end{subfigure}
\caption{Templates for proving Tightness~\ref{obs:disj:tight_prop}}
\label{fig:obs:disj:tight_prop}
\end{figure}

We need one process in every state of $B$ to enable the transitions of $A$ to ${all}_A$. Only when $A$ in ${all}_A$, $B_1$ can move $3_B \to 1_B$, and then at some point to $2_B$. After $B_1$ moves $3_B \to 1_B$, $A$ moves ${all}_A \to {end}_A$, which requires process $B_{i \neq 1}$ in $3_B$. Finally, to make the run infinite, there should be at least two processes in the state $k_B$.
Hence, every infinite run satisfying the formula needs at least $|B|+2$ $B$-processes.
\sj{other cases are covered in general lemma below}
% 
\end{proof}


\subsection{\LTLmX\ Properties with Fairness: New Constructions} \label{gua:sec:ideas-disj-fair}

As for the case without fairness, proving the monotonicity lemma is simple.

\begin{lemma}[Monotonicity: Disj, \LTLmX, Fair] \label{disj:le:FairDisjunctiveMonotonicity}
For disjunctive systems:
\begin{align*}
& \forall n \geq 1: \\
&(A,B)^{(1, n)} \models \pexists_{uncond} h(A,B_1) 
\implies
(A,B)^{(1,n+1)} \models \pexists_{uncond} h(A,B_1),\\
\end{align*}
\end{lemma}
\begin{proof}
In run $x$ of $(A,B)^{(1,n)}$ with $n \geq 1$ all processes move infinitely often. 
Hence let the run $y$ of $(A,B)^{(1,n+1)}$ copy $x$, 
and let the new process mimic an infinitely moving B process of $(A,B)^{(1,n)}$.
\end{proof}

To prove the bounding lemma, we introduce two new constructions.
We need new constructions, because the flooding construction does not preserve fairness,
and also cannot be used to construct deadlocked runs,
since it does not preserve disabledness of transitions of processes $A$ or $B_1$.

Consider the proof task of the bounding lemma for disjunctive systems with fairness:
given an unconditionally fair run $x$ of 
\largesys with 
$x \models h(A,B^{(1)})$, we want to construct an unconditionally fair run $y$ 
of \cutoffsys with $y \models h(A,B^{(1)})$. In contrast to unfair systems, we 
need to ensure that all processes move infinitely often in $y$. 
The insight is 
that after a finite time all processes will start looping 
around some set $\visited^\inf$ of states. We construct a run $y$ that
mimics this. To this end, we introduce two constructions. \emph{Flooding with
evacuation} is similar to flooding, but instead of keeping
processes in their flooding states forever it evacuates the processes into 
$\visited^\inf$. \emph{Fair extension} lets all processes move infinitely 
often without leaving $\visited^\inf$.

\myparagraph{Flooding with evacuation}
Given a subset $\mF \subseteq \mB$
%\footnote{In this section we will use only $\mP_1 = \{B_1\}$, 
%          but for the case of deadlocks we will need a set.  
%          $\mP_1$ is a set of processes whose local runs we will copy 
%          from $x$ to $y$ later in the proof.} 
and an infinite run $x=(s_1,e_1,p_1)\ldots$ of \largesys, 
define
\begin{align}
& \visInf{\mF}{x} = \{ q \|\! \exists \text{ infinitely many } ~~~~ m\!:  
s_m(B_i) = q 
\text{ for some } B_i \in \mF \} \label{disj:def_vinf_wrt} \\
& \visFin{\mF}{x} = \{ q \|\! \exists \text{ only finitely many } m\!:  
s_m(B_i) = q
\text{ for some } B_i\in \mF \} \label{disj:def_vfin_wrt}
\end{align}
%
Let $q \in \visFin{\mF}{x}$.
In run $x$ there is a moment $f_q$ when $q$
is reached for the first time by some process from $\mF$, denoted $B_{\first_q}$. 
Also, in run $x$ there is a moment $l_q$ such that:
$s_{l_q}(B_{\last_q})=q$ for some process $B_{\last_q} \in \mF$, 
and $s_t(B_i)\neq q$ for all $B_i \in \mF$,
$t > l_q$%
---i.e., when some process from $\mF$ is in state $q$ for the last time in $x$. 
Then, saying that process $B_{i_q}$ of \cutoffsys\ 
{\em floods $q \in \visFin{\mF}{x}$ and then evacuates into $\visInf{\mF}{x}$} 
means: 
$$
y(B_{i_q}) = x(B_{\first_q})\slice{1}{f_q} \ \cdot\ (q)^{(l_q - f_q + 1)} \cdot \ 
x(B_{\last_q})\slice{l_q}{m} \ \cdot \ (q')^\omega,
$$
where $q'$ is the state in $\visInf{\mF}{x}$ that $x(B_{\last_q})$ reaches first, 
at some moment $\time \geq l_q$.
In words, process $B_{i_q}$ mimics process $B_{\first_q}$ until it reaches $q$, 
then does nothing until process $B_{\last_q}$ starts leaving $q$, 
then it mimics $B_{\last_q}$ until it reaches $\visInf{\mF}{x}$.

The construction ensures: 
if we copy local runs of all processes not in $\mF$ from $x$ to $y$, 
then all transitions of $y$ are enabled. 
This is because, for any process $p$ of $\cutoffsys$ that takes a transition in $y$ at any moment, 
the set of states visible to process $p$ is a superset of the set of states 
visible to the original process in \largesys whose transitions process $p$ copies.


\myparagraph{Fair extension} 
\ak{explain intuition about those three sets}
\ak{adapt to dead}
Here, we consider a path $x$ that is the postfix of an unconditionally fair run $x'$ of $\largesys$, 
starting from the moment where no local states from $\visFin{\mB}{x'}$ are visited anymore. 
We construct a corresponding unconditionally-fair path $y$ of $\cutoffsys$, 
where no local states from $\visFin{\mB}{x'}$ are visited.

Formally, let $n \geq 2|B|$, and $x$ an unconditionally-fair path of $\largesys$ such that
$\visFin{\mB}{x}=\emptyset$.
Let $c \geq 2|B|$, and $s_1'$ a state of \cutoffsys
with
\li
\- $s_1'(A_1)=s_1(A_1)$, $s_1'(B_1)=s_1(B_1)$;

\- for every $q \in \visInf{B_2..B_n}{x} \smi \visInf{B_1}{x}$,
   there are two processes $B_{i_q}, B_{i_q'}$ of \cutoffsys
   that start in $q$, i.e., $s_1'(B_{i_q})=s_1'(B_{i_q'})=q$;

\- for every $q \in \visInf{B_2..B_n}{x} \cap \visInf{B_1}{x}$,
   there is one process $B_{i_q}$ of \cutoffsys
   that starts in $q$;

\- for some $\qstar \in \visInf{B_2..B_n}{x} \cap \visInf{B_1}{x}$,
   there is one additional process of \cutoffsys, 
   different from any in the above, 
   called $B_{i_\qstar'}$,
   that starts in $\qstar$;
   and

\- any other process $B_i$ of \cutoffsys 
   starts in some state of $\visInf{B_2..B_n}{x}$.
\il
Note that, if $\visInf{B_2..B_n}{x}\cap \visInf{B_1}{x} = \emptyset$, 
then the third and fourth pre-requisite are trivially satisfied.

The fair extension extends state $s_1'$ of \cutoffsys 
to an unconditionally-fair path $y=(s'_1,e'_1,p'_1)\ldots$ 
with $y(A_1,B_1) = x(A_1,B_1)$ as follows.
\li
\-[(a)] $y(A_1)=x(A_1)$, $y(B_1)=x(B_1)$.

\-[(b)] For every $q \in \visInf{B_2..B_n}{x} \smi \visInf{B_1}{x}$: 
       in run $x$ there is $B_i \in \{B_2..B_n\}$ 
       that starts in $q$ and visits it infinitely often. 
       Let $B_{i_q}$ and $B_{i'_q}$ of \cutoffsys mimic $B_i$ in turns: 
       first $B_{i_q}$ mimics $B_i$ until it reaches $q$, 
       then $B_{i'_q}$ mimics $B_i$ until it reaches $q$, and so on.

\-[(c)] Arrange the states of $\visInf{B_2..B_n}{x}\cap \visInf{B_1}{x}$ 
       in some order $(\qstar, q_1, \ldots, q_l)$.  
       The processes $B_{i_\qstar'}, B_{i_\qstar}, B_{i_{q_1}}, \ldots, B_{i_{q_l}}$ 
       behave as follows.
       Start with $B_{i_\qstar'}$: 
       when $B_1$ enters $\qstar$ in $y$, it carries%
       \footnote{``Process $B_1$ starting at moment $m$ carries process $B_i$ 
                 from $q$ to $q'$'' means: process $B_i$ mimics 
                 the transitions of $B_1$ starting at moment $m$ at $q$ 
                 until $B_1$ first reaches $q'$.}
       $B_{i_\qstar'}$             from $\qstar$ to $q_1$, 
       then carries $B_{i_{q_1}}$ from $q_1$ to $q_2$, \ldots, 
       then carries $B_{i_{q_l}}$ from $q_l$ to $\qstar$, 
       then carries $B_{i_\qstar}$ from $\qstar$ to $q_1$, 
       then carries $B_{i_\qstar'}$ from $q_1$ to $q_2$, 
       then carries $B_{i_{q_1}}$ from $q_2$ to $q_3$,
       and so on.

%\-[c2.] otherwise, $\visited_{\inf\cap B_1}{x} = \{ q^\star \}$. Then $B_1$ only ever makes transitions $q^\star \to q^\star$, thus let process $B_{i_{q^\star}}$ mimic this.

\-[(d)] Any other $B_i$ of \cutoffsys,
       starting in $q \in \visInf{B_2..B_n}{x}$,
       mimics $B_{i_q}$.
\il
Note that parts (b) and (c) of the construction ensure that there is always at
       least one process in every state from $\visInf{B_2..B_n}{x}$. This
       ensures that the guards of all transitions of the construction are satisfied.
Excluding processes in (d), the fair extension uses up to $2|B|$ copies of $B$.%
\footnote{A careful reader may notice that,
          if
          $|\visInf{B_1}{x}|=1$ and $|\visInf{B_2..B_n}{x}|=|B|$,
          then the construction uses $2|B|+1$ copies of $B$.
          But one can slightly modify the construction for this special case,
          and remove process $B_{i_\qstar'}$ from the pre-requisites.}
%\li
%\- if $\visInf{B_2..B_n}{x} \cap \visInf{B_1}{x} = \emptyset$, then 
%     $$\leq 1+2|\visInf{B_2..B_n}{x}| \leq 1+2(|B|-1) = 2|B|-1$$
%   (note that $\visInf{B_1}{x}$ contains at least one state)
%
%\- otherwise:\\
%   let $smi = \visInf{B_2..B_n}{x} \smi \visInf{B_1}{x}$, \\
%   let $inter = \visInf{B_2..B_n}{x} \cap \visInf{B_1}{x}$, \\
%   then
%   $$\leq 1+2|smi| + |inter| + 1 \leq 1+2(|B|-1) + 1 + 1 = 2|B|+1$$
%   $$\leq 1+2|smi| + |inter| + 1 \leq 1+2(|B|-2) + 2 + 1 = 2|B|+1$$
%\il
%
%
%%%%%%%%%%%%% OLD FAIR EXTENSION %%%%%%%%%%%%%%%%%%%%%%
%Let $x=(s_1,e_1,p_1)\ldots$ be an unconditionally-fair path $x$ of $\largesys$, let $\visited_\fin{x}$ and $\visited_\inf{x}$ be defined wrt. no processes, let $n\geq 2|\visited_\inf{x}|$, and let $x$ satisfy: 
%\li
%  \- every $B_i\neq B_1$ visits $s_1(B_i)$ infinitely often
%  \- $\visited_\fin{x} = \emptyset$,
%\il
%and let $s_1'$ be a state of \cutoffsys with $c \geq 2|\visited_\inf{x}|$ that satisfies: 
%\li
%  \- $s_1'(A)=s_1(A)$, $s_1'(B_1)=s_1(B_1)$
%  \- for every $q \in \visited_\inf{x}$ there are two processes of \cutoffsys called $B_{i_q}$, $B_{i'_q}$ with $s_1'(B_{i_q}) = s_1'(B_{i'_q}) = q$
%  \- for all other processes $B_i$ of \cutoffsys (if any): $s_1'(B_i) \in \visited_\inf{x}$.
%\il
%The fair extension extends state $s_1'$ of \cutoffsys to an unconditionally-fair path $y=(s'_1,e'_1,p'_1)\ldots$ with $x(A,B_1) = y(A,B_1)$ as follows. Let $\visited_{\inf\cap B_1}{x}$ be the set of states visited infinitely often by process $B_1$, and $q^\star=s_1(B_1)$:
%\li
%\-[a.] $y(A)=x(A)$, $y(B_1)=x(B_1)$
%\-[b.] for every $q \in \visited_\inf{x} \smi \visited_{\inf\cap B_1}{x}$: in run $x$ there is $B_i$ that starts in $q$ and visits it infinitely often. Let $B_{i_q}$ and $B_{i'_q}$ of \cutoffsys mimic $B_i$ in turns: first $B_{i_q}$ mimics $B_i$ until it reaches $q$, then $B_{i'_q}$ mimics $B_i$ until it reaches $q$,\dots 
%\-[c1.] if $\visited_{\inf\cap B_1}\smi \{ q^\star \} \neq \emptyset$:
%\li
%  \- order arbitrarily $\visited_{\inf\cap B_1}\smi \{ q^\star \} = (q_1, q_2, \ldots, q_k)$
%  \- the processes $\{ B_{i_{q^\star}}, B_{i_{q_1}}, B_{i'_{q_1}}, B_{i_{q_2}}, B_{i'_{q_2}}, \ldots, B_{i_{q_k}}, B_{i'_{q_k}} \}$ behave as follows:
%  \- start with $B_{i_{q_1}}$: when $B_1$ enters $q_1$, it carries $B_{i_{q_1}}$ from $q_1$ to $q_2$, then carries $B_{i_{q_2}}$ from $q_2$ to $q_3$, \ldots, then carries $B_{i_{q_k}}$ from $q_k$ to $q^\star$, then carries $B_{i_{q^\star}}$ from $q^\star$ to $q_1$, then carries $B_{i'_{q_1}}$ from $q_1$ to $q_2$, and so on.
%\il
%\-[c2.] otherwise, $\visited_{\inf\cap B_1} \!=\! \{ q^\star \}$: $B_1$ only transits $q^\star \!\to\! q^\star$; let $B_{i_{q^\star}}$ mimic $B_1$
%
%\-[d.] let other processes $B_i$ of \cutoffsys with $s_1'(B_i)=q$ (if any) mimic $B_{i_q}$.
%\il
%%%%%%%%%%%%% END OF OLD FAIR EXTENSION %%%%%%%%%%%%%%%%%%%%%%

%\begin{proof}[Proof idea of the bounding lemma]

Now we are ready to prove the bounding lemma.

\begin{lemma}[Bounding: Disj, \LTLmX, Fair] \label{disj:le:FairDisjunctiveBounding}
For disjunctive systems:
\begin{align*}
&\forall n>2|B|: \\
&(A,B)^{(1,2|B|)} \models \pexists_{uncond} h(A,B_1) &
&\impliedby& &
(A,B)^{(1,n)} \models \pexists_{uncond} h(A,B_1),\\
\end{align*}
% 
\end{lemma}
\begin{proof}
\sj{for weak or strong fairness, the same construction can be used; evacuation is not necessary, but also doesn't increase the cutoff if we use it; difficulty: show that cutoff is still tight
}
Let $c=2\card{B}$. 
Given an unconditionally-fair run $x$ of $\largesys$,
we construct an unconditionally-fair run $y$ of the cutoff system $\cutoffsys$ 
such that $y(A,B_1)$ is stuttering equivalent to $x(A,B_1)$.

Note that in $x$ there is a moment $m$ such that all local states that are visited after $m$ are in $\visInf{\mB}{x}$.

The construction has two phases. In the first phase, we apply flooding for states in $\visInf{\mB}{x}$, and flooding with evacuation for states in $\visFin{\mB}{x}$:
\li
\-[(a)] $y(A)=x(A)$, $y(B_1)=x(B_1)$;

\-[(b)] for every $q \in \visInf{B_2..B_n}{x} \smi \visInf{B_1}{x}$, 
       devote two processes of $\cutoffsys$ that flood $q$;

\-[(c)] for some $\qstar \in \visInf{B_2..B_n}{x} \cap \visInf{B_1}{x}$,
       devote one process of \cutoffsys that floods $\qstar$;

\-[(d)] for every $q \in \visFin{B_2..B_n}{x}$, 
       devote one process of $\cutoffsys$ that 
       floods $q$ and evacuates into $\visInf{B_2..B_n}{x}$;
       and

\-[(e)] let other processes (if any) mimic process $B_1$.
\il
The phase ensures that at moment $m$ in $y$, 
there are no processes in $\visFin{\mB}{x}$, 
and all the pre-requisites of the fair extension are satisfied.

The second phase applies the fair extension, 
and then establishes the interleaving semantics 
as in the bounding lemma in the non-fair case.
The overall construction uses up to $2|B|$ copies of $B$.
% Indeed: 
% fin&smi=0, fin&cap=0,
% smi = InfB2..Bn - cap
% InfB2..Bn <= B - fin
% Then:
% 1+2smi+1cap+1+fin <= 1+2(InfB2..Bn-cap)+cap+1+fin = 
%                      2+2InfB2..Bn-cap-fin <=
%                      2+2B-cap-fin
% now split case:
% everywhere fin>0 (otherwise 2|B| follows from the analysis of the fair extension)
% note: if cap=0, then we actually have (recall fair pre)
%       1+2smi+fin =< 1+2(B-InfB1-fin)+fin = 1+2B-2InfB1-2fin =< 2B-3
% thus cap>0,fin>0
% then 2+2B-cap-fin =< 2B
\end{proof}

\begin{tightness}[Disj, \LTLmX, Fair] \label{obs:disj:fair_tight_prop}
The cutoff in Lemma~\ref{disj:le:FairDisjunctiveBounding} is tight.
I.e., 
for any $k$ there exist process templates $(A,B)$ with $|B| = k$ 
and $\LTLmX$ formula $h(A,B_1)$ such that:
$$
(A,B)^{(1,2|B|)} \models \pexists h(A,B_1) ~~and~~ 
(A,B)^{(1,2|B|-1)} \not\models \pexists h(A,B_1).
$$
\ak{what happens if we bound $T_A$?}
\end{tightness}
\begin{proof}
Consider process templates $A,B$ from Figure~\ref{fig:obs:disj:fair_tight_prop}
and the property $\pexists \true$.
% 
\begin{figure}[tb]
\centering
\begin{subfigure}[b]{0.35\textwidth}\center
\scalebox{0.75}{%!TEX root = table.tex
\begin{tikzpicture}[node distance=2cm,inner sep=1pt,minimum size=0.5mm,->,>=latex]

\node[state,initial left] (a_1) {$1_A$};
\node (dots) [right of=a_1] {$\ldots$};
\node[state] (a_all) [right of=dots] {${all}_A$};

\path (a_1) edge [above] node {$\disj{1_B}$} (dots);
\path (dots) edge [above] node {$\disj{k_B}$} (a_all);
\path (a_all) [loop above] edge node {} (a_1);

\end{tikzpicture}
}
\caption*{Template A}
\end{subfigure}
\hspace{1cm}
\begin{subfigure}[b]{0.55\textwidth}\center
\scalebox{0.75}{%!TEX root = table.tex
\begin{tikzpicture}[node distance=2.4cm,inner sep=1pt,minimum size=0.5mm,->,>=latex]


\node[state,initial left] (b_1) {$1_B$};
\node[state] (b_2) [right of=b_1] {$2_B$};
\node (dots) [right of=b_2] {$\ldots$};
\node[state] (b_k) [right of=dots] {$k_B$}; 

\path (b_1) edge [above] node {$\disj{1_B}$} (b_2);
\path (b_2) edge [above] node {$\disj{2_B}$} (dots);
\path (dots) edge [above] node {$\disj{{k-1}_B}$} (b_k);

\path (b_1) [loop above] edge [right] node {$\disj{1_B}$} (b_1);
\path (b_2) [loop above] edge [right] node {$\disj{2_B}$} (b_2);
\path (b_k) [loop above] edge [right] node {$\disj{k_B}$} (b_k);


\end{tikzpicture}
}
\caption*{Template B}
\end{subfigure}
\caption{Templates for proving Tightness~\ref{obs:disj:fair_tight_prop}}
\label{fig:obs:disj:fair_tight_prop}
\end{figure}
%
\end{proof}


\subsection{Deadlocks without Fairness: Updated Constructions} \label{gua:sec:proofs-disj-deadlock-unfair}

%The lemma for deadlock detection, for fair and unfair cases,
%is proven for $n \geq |B|+1$.
%In the case of local deadlocks, 
%process $B_{n+1}$ mimics a process that moves infinitely often in $x$.
%In the case of global deadlocks, 
%by pigeon hole principle, 
%in the global deadlock state there is a state $q$ with at least two processes in it---let process $B_{n+1}$ mimic a process that deadlocks in $q$.

\begin{lemma}[Monotonicity: Disj, Deadlocks, Unfair]
\label{mono_lem_disj_deadlocks_unfair}
    For disjunctive systems:
    $$\forall n\geq |B|+1: (A,B)^{(1,n)} \textit{ has a deadlock} \ 
    \Impl\ 
    (A,B)^{(1,n+1)} \textit{ has a deadlock.}$$
\end{lemma}
\begin{proof}
Given a deadlocked run $x$ of $(A,B)^{(1,n)}$,
we build a deadlocked run of $(A,B)^{(1,n+1)}$. 
If the run $x$ is locally deadlocked,
then it has at least one infinitely moving process, 
thus let the additional process mimic that process. 
If the run $x$ is globally deadlocked run, 
then due to $n>|B|$ in some state there are at least two processes deadlocked. 
Thus, let the new process mimic a process deadlocked in that state---%
the run constructed will also be globally deadlocked.
\end{proof}


\begin{lemma}[Bounding: Disj, Deadlocks, Unfair] \label{lem_disj_deadlocks_unfair}
For disjunctive systems:
\li
  \- with $c=|B|+2$ and any $n>c$:
  $$(A,B)^{(1,c)} \textit{ has a local deadlock} \ \Implied\ (A,B)^{(1,n)} \textit{ has a local deadlock;}$$
  
  \- with $c=2|B| - 1$ and any $n>c$
  $$(A,B)^{(1,c)} \textit{ has a global deadlock} \ \Implied\ (A,B)^{(1,n)} \textit{ has a global deadlock;} $$
  
  \- with $c=2|B|-1$ and any $n>c$:
  $$(A,B)^{(1,c)} \textit{ has a deadlock} \ \Implied\ (A,B)^{(1,n)} \textit{ has a deadlock.}$$
\il
\ak{seems not tight}
\end{lemma}
\begin{proof}[Proof idea]
First, consider the case of global deadlocks.
The insight is to divide deadlocked local states into two disjoint sets, 
$\dead_1$ and $\dead_2$, as follows.
Given a globally deadlocked run $x$ of \largesys, 
for every $q \in \dead_1$,
there is a process of \largesys deadlocked in $q$ with input $i$,
that has an outgoing transition guarded ``$\exists q$''%
---hence, adding one more process into $q$ would unlock the process.
%\sj{do we always consider inputs correctly? what if $q \in \dead_1$ for some $e$, but $q \in \dead_2$ for $e'$?}\ak{thanks, modified, now it is impossible}
In contrast, $q \in \dead_2$ if any process deadlocked in $q$
stays deadlocked after adding more processes into $q$.
Let us denote the set of $B$-processes deadlocked in $\dead_1$ by $\mD_1$.
Finally, abuse the definition in Eq.~\ref{disj:def_vfin_wrt}
and denote by $\visFin{\mB\smi\mD_1}{x}$ the set of states
that are visited by $B$-processes not in $\mD_1$ before reaching a deadlocked state.

Given a globally deadlocked run $x$ of \largesys with $n\geq 2|B|-1$, 
we construct a globally deadlocked run $y$ of \cutoffsys with $c = 2|B|-1$ as follows.
\li
\- We copy from $x$ into $y$ the local runs of processes in $\mD_1 \cup \{A\}$;
\- flood every state of $\dead_2$;
   and
\- for every $q \in \visFin{\mB\smi\mD_1}{x}$,
   flood $q$ and evacuate into $\dead_2$.
\il
The construction ensures: 
(1) for any moment and any process in $y$,
    the set of local states that are visible to the process includes all the states that were visible 
    to the corresponding process in \largesys whose transitions we copy;
(2) in $y$, there is a moment when all processes deadlock in $\dead_1 \cup \dead_2$.

For the case of local deadlocks, 
the construction is slightly more involved,
since we also need to copy the behaviour of an infinitely moving process.
\end{proof}

\begin{proof}
Given a (globally or locally) deadlocked run of $\largesys$,
we construct (globally or locally) deadlocked run of $\cutoffsys$, 
where $c$ depends on the nature of the given run. 
We do this using the construction template. 

Let $\mB=\{B_1,...,B_n\}$.
The template depends on the set $\mC \subseteq \{B_1,...,B_c\}$
and is as follows.
\li
  \-[a.] Set $y(A)=x(A)$;
  \-[b.] for every $B_i \in \mC$, set $y(B_i)=x(B_i)$;
  \-[c.] for every $q \in \visInf{\mB\smi\mC}{x}$, 
         devote one process of \cutoffsys that floods $q$;
  \-[d.] for every $q \in \visFin{\mB\smi\mC}{x}$, 
         devote one process of \cutoffsys that floods $q$ 
         and then evacuates into $\visInf{\mB\smi\mC}{x}$;
         and
  \-[e.] let other processes (if any) mimic some process from (c).
\il

\myparagraph{1) Local deadlock}
We distinguish three cases: 
\li
  \-[1a)] $A$ deadlocks, $B_1$ moves infinitely often;
  \-[1b)] $A$ moves infinitely often, $B_1$ deadlocks; and
  \-[1c)] $A$ neither deadlocks nor moves infinitely often, 
          $B_1$ deadlocks, $B_2$ moves infinitely often.
\il

\myparagraphraw{1a:} ``$A$ deadlocks, $B_1$ moves infinitely often''. 

Let $c=|B|+1$, and $\mC=\{B_1\}$.
Note that $\visInf{B_2..B_n}{x} \neq \emptyset$. 
The resulting construction uses 
$|\visFin{B_2..B_n}{x}| + |\visInf{B_2..B_n}{x}| + 1 
 \leq 
 |B| + 1$ 
copies of B.
\ak{seems tight}\ak{correctness}

\myparagraphraw{1b:} ``$A$ moves infinitely often, $B_1$ deadlocks''. 

Let $c=|B|+1$, and $\mC=\{B_1\}$.
Let $q_\bot$ be the state in which $B_1$ deadlocks.
Instantiate the construction template.

Process $B_1$ of \cutoffsys is deadlocked in $y$ starting from some moment $d$,
because any state it sees (in $\visInf{A,B_2..B_n}{x}$)
was also seen by $B_1$ in \largesys in $x$ at some moment $d' \geq d$
(note that $d'$ may be not the same moment as $d$).
%\footnote{Note about open systems: here we use the fact from the definitions 
%          that inputs to $B_1$ do not change.
%          This ensures that the set of states that $B_1$ should not see in order
%          to stay deadlocked does not change over time.}


\myparagraphraw{1c:} ``$A$ neither deadlocks nor moves infinitely often, 
                       $B_1$ deadlocks, $B_2$ moves infinitely often''. 

Instantiate the construction template with $c=|B|+2$ and $\mC = \{B_1,B_2\}$.
\ak{seems not tight}\ak{correctness}

\smallskip
Finally, $|B|+2$ is a (possibly not tight) cutoff for local deadlock detection problem.


\myparagraph{2) Global deadlock}
Let $x=(s_1,e_1,p_1)...(s_d,e_d,\bot)$ be a globally deadlocked run of $\largesys$ 
with $n\geq c$.

Let us abuse the definition of $\visInf{\mF}{x}$ and $\visFin{\mF}{x}$,
in Eq.~\ref{disj:def_vinf_wrt} and \ref{disj:def_vfin_wrt} resp., 
and adapt it to the case of finite runs.
To this end, given a finite run $x=(s_1,e_1,p_1)...(s_d,e_d,\bot)$, 
extend it to the infinite sequence $(s_1,e_1,p_1)...(s_d,e_d,\bot)^\omega$, 
and apply the definition of $\visInf{\mF}{x}$ and $\visFin{\mF}{x}$ to the sequence.

Let $\mD_1$ be the set of processes deadlocked in unique states:
$\forall p\in \mD_1 \nexists p' \neq p: s_d(p')=s_d(p)$.
Instantiate the construction template with $\mC = \mD_1$ and $c=2|B|-1$.
\footnote{$2|B|-1$ copies is enough, because: 
          $\visFin{\mB\smi\mC}{x} \cap \visInf{\mB\smi\mC}{x} = \emptyset$,
          $\visInf{\mB\smi\mC}{x} \cap \visInf{\mC}{x} = \emptyset$,
          and if $\visFin{\mB\smi\mC}{x} \neq \emptyset$, 
          then $\visInf{\mB\smi\mC}{x} \neq \emptyset$.}
\ak{seems not tight}

%The construction uses $|dead1| + |dead2| + |\visited_{\fin-P_\bot^1}(x)| \leq 2|B|-1$ copies of B.\ak{seems not tight}\ak{CHECK}\ak{correctness}

\myparagraph{3) Deadlocks}
As the cutoff for the deadlock detection problem we take the largest cutoff in (1)--(2), namely, $2|B|-1$,
but it may be not tight%
---finding the tight cutoffs for local deadlock and for deadlock detection problems is an open problem.

\ak{tried to refine but could not -- the trial is commented out}
% AK: try to utilize the existence of the order?
% Now let us refine the estimate:
% \li
% \- `unique finite state' is any state in $\visited_{\fin+P_1_\bot}$ that either is not in $ \visited_{\fin-P_1_\bot} $, or the state that appears not in $]f_q,l_q[_{-P_1_\bot}$
% 
% \- intuitively, state is unique finite if we cannot make it `visible' by using the flooding and evacuating construction wrt. $ \visited_{\fin-P_1_\bot} $, $\visited_{\inf-P_1_\bot}$. Otherwise, if state is not unique finite then  we can flood the state and then evacuate into $ \visited_{\inf-P_1_\bot} $ 
% 
% \- let $U^c \subseteq P_1_\bot$ be the set of processes that have unique finite states on its run to $q_\bot \in \visited_\bot^1$. Let $U^f = P_1_\bot\smi U^c$.
% 
% \- note that for any $p \in U^f$ any its finite state can be flooded and then evacuated into $ \visited_{\inf-P_1_\bot} $ using local runs of $\mathcal{B}\smi P_1_\bot$ processes. Intuitively, this means that $U^f$ processes' finite states are not needed.
% 
% \- let us divide $\visited_\bot^1$ into two disjiont sets: $\visited_\bot^c$ (and corr. processes are $U^c$) is the set of states whose deadlocked paths do have a unique finite state, and $\visited_\bot^f$ (and corr. processes are $U^f$) is the set of states whose deadlocked paths do not have a unique finite state.
% \il
% 
% Define $ \visited_{\fin-U^c} $ and $ \visited_{\inf-U^c} $ wrt. $U^c$.
% Then the construction is:
% \li
%   \-[a.] $y(A)=x(A)$
%   \-[b.] copy the runs of $U^c$ processes (that end in $\visited_\bot^c$)
%   \-[c.] flood states in $ \visited_{\inf-U^c} $
%   \-[d.] for any $ q \in \visited_{\fin-U^c} $ devote one process that floods it and then evacuates into $ \visited_{\inf-U^c} \smi \visited_\bot^f$. Note that we can evacuate a state from $ \visited_{\fin-U^c} $ into $ \visited_{\inf-U^c} \smi \visited_\bot^f $, because processes $U^f$ do not contribute to the evacuation by definition.\ak{vague}
%   \-[e.] let other processes (if any) mimic a process from (c)
% \il
% The construction uses $|\visited_\bot^c| + | \visited_{\inf-U^c} | + | \visited_{\fin-U^c} |$. Note that:
% \li
%   \- $\visited_\bot^c$, $ \visited_{\inf-U^c} $ are disjoint
%   \- $\visited_{\inf-U^c}$, $\visited_{\fin-U^c}$ are disjoint
%   \- possibly $\visited_\bot^c \cap \visited_{\fin-U^c} \neq \emptyset$
% \il 
% If some $p \in U^c$, then it has at least one unique finite state $q$: possibly $q \in  \visited_{\inf-U^c} $ or $q \in \visited_{\fin-U^c}$, but $q \not\in \visited_\bot^c$. \ak{stuck here -- it is possible that all $U^c$ processes share a single unique finite state}
\end{proof}

\begin{tightness}[Disj, Deadlocks, Unfair] \label{obs:disj:tight_deadlock}
The cutoff $c=2|B|-1$ for deadlock detection in disjunctive systems is \emph{asymptotically optimal but possibly not tight}.
I.e., for any $k$ there are templates $(A,B)$ with $|B|=k$ such that:
$$
(A,B)^{(1,|B|-1)} \textit{ does not have a deadlock, but } (A,B)^{(1,|B|)} \textit { does}.
$$
\end{tightness}
\begin{proof}
Figure~\ref{fig:obs:disj:tight_deadlock} illustrates templates $(A,B)$ to prove the asymptotic optimality of cutoff $2|B|-1$ for deadlock detection problem. Template $A$ is any that never deadlocks. The system has a local deadlock only when there are at least $|B|$ copies of $B$, which is a constant factor of $2|B|-1$.
\begin{figure}[tb] \centering
\makebox[0.4\textwidth][c]{
\scalebox{0.75}{\begin{tikzpicture}[node distance=2.4cm,inner sep=1pt,minimum size=0.5mm,->,>=latex]

\node[initial left, state] (b_1) {$1_B$};
\node[state] (b_2) [right of=b_1] {$2_B$};
\node (dots) [right of=b_2] {$\ldots$};
\node[state] (b_k) [right of=dots] {$k_B$}; 

\path (b_1) edge [above] node {$\disj{1_B}$} (b_2);
\path (b_2) edge [above] node {$\disj{2_B}$} (dots);
\path (dots) edge [above] node {$\disj{{k-1}_B}$} (b_k);

\path (b_1) [loop above] edge [right] node {} (b_1);
\path (b_2) [loop above] edge [right] node {} (b_2);
% \path (b_k) [loop above] edge [right] node {$\disj{b_k}$} (b_k);

\path (b_1) edge [below, bend right=22] node {} (b_k);
\path (b_2) edge [below, bend right=15] node [near start] {$\disj{{k}_B}$} (b_k);

\end{tikzpicture}



% \begin{tikzLTS}
% \tikzstyle{state} = [circle,draw=black,thick,inner sep=1.5pt, text width={width("$\text{init}_2$")}, align=center]
% \tikzstyle{boxLabel} = [yshift=-0.4cm]
% \tikzstyle{every node} = [node distance=1.75cm]
% 
% \node[state] (t1) {$b_1$};
% \node[state] (t2) [below of=t1] {$b_2$};
% \node[] (t3) [below of=t2] {$$};
% \node[] (t4) [below of=t3, node distance=1cm] {$$};
% \node[state] (t5) [below of=t4] {$b_k$};
% 
% \path (t1) [post,loop right] edge node [right] (l1) {} (t1);
% \path (t1) [post] edge node[right] {$\disj{b_1}$} (t2);
% \path (t2) [post,loop right] edge node [right] (l2) {} (t2);
% \path (t2) [post,dashed] edge node[right] {$\disj{b_2}$} (t3);
% \path (t3) [dotted] edge (t4);
% \path (t4) [post,dashed] edge node[right] {$\disj{b_{k-1}}$} (t5);
% \path (t5) [post,loop right] edge node[right] (l5) {$\disj{b_k}$} (t5);
% 
% \node (U2) [draw=black, fit=(t1) (t2) (t3) (t4) (t5) (l1) (l2) (l5), inner sep=0.75cm] {} ;
% \node [boxLabel] at (U2.north) {Template $B$};
% 
% \end{tikzLTS}
}
}
\caption{Templates for proving Tightness~\ref{obs:disj:tight_deadlock}}
\label{fig:obs:disj:tight_deadlock}
\end{figure}
\end{proof}

%--- global deadlocks: fair and unfair:
%C: processes that dead1
%F: processes that dead2
%We copy local runs of dead1.
%We flood deadlocked states of dead2, and flood and evacuate non-deadlocked states of dead2.
%
%--- local deadlocks: unfair:
%I: processes that move infinitely often
%D: processes that dead
%copy local run of one process from I, 
%copy one local run of process from C,
%flood and evacuate finitely visited states by processes except copied
%
%--- local deadlocks: fair:
%I: processes that move infinitely often
%C: processes that dead1
%F: processes that dead2
%Copy local runs of C, 
%flood and evacuate finitely visited states of F\\C, 
%flood dead or infinitely often visited states of F\\C.


\subsection{Deadlocks with Fairness: New Constructions} \label{gua:sec:proofs-disj-deadlock-fair}

\begin{lemma}[Monotonicity: Disj, Deadlocks, Fair] \label{mono_lem_disj_deadlocks_fair}
For disjunctive systems, on strong-fair or finite runs:
$$
\forall n\geq |B|+1: (A,B)^{(1,n)} \textit{ has a deadlock} 
\ \Impl\ 
(A,B)^{(1,n+1)} \textit{ has a deadlock.}
$$
\end{lemma}
\begin{proof}
See proof of Lemma~\ref{mono_lem_disj_deadlocks_unfair}.
\end{proof}

\begin{lemma}[Bounding: Disj, Deadlocks, Fair] \label{le:disj:fair_tight_deadlock}
For disjunctive systems, on strong-fair or finite runs:
\li
  \- with $c=2|B|-1$ and any $n>c$:
  $$(A,B)^{(1,c)} \textit{ has a local deadlock} \ \Implied\ (A,B)^{(1,n)} \textit{ has a local deadlock;}$$
  
  \- with $c=2|B| - 1$ and any $n>c$
  $$(A,B)^{(1,c)} \textit{ has a global deadlock} \ \Implied\ (A,B)^{(1,n)} \textit{ has a global deadlock;} $$
  
  \- with $c=2|B|-1$ and any $n>c$:
  $$(A,B)^{(1,c)} \textit{ has a deadlock} \ \Implied\ (A,B)^{(1,n)} \textit{ has a deadlock.}$$
\il
\end{lemma}
The proofs are similar to that of Lemma~\ref{lem_disj_deadlocks_unfair} (the case without fairness):
the case of global deadlocks is exactly the same,
the case of local deadlocks differ---we additionally use the fair extension to ensure the resulting run is fair.
\begin{proof}

\providecommand{\deadOne}[1]{\dead_{<2}(#1)}
\providecommand{\deadTwo}[1]{\dead_2(#1)}

If $\largesys$ has a global deadlock, 
then the fairness does not influence the cutoff, 
and the proof from Lemma~\ref{lem_disj_deadlocks_unfair}, 
case ``Global Deadlocks'', applies and gives the cutoff $2|B|-1$. 
Hence below consider only the case of local deadlocks. 

Given a strong-fair deadlocked run $x$ of $\largesys$, 
we first construct a strong-fair deadlocked run $y$ of $\cutoffsys$ 
with $c=2|B|$ and then argue that $c$ can be reduced to $2|B|-1$. 
The construction is similar to that in Lemma~\ref{lem_disj_deadlocks_unfair} 
-- the differences originate from the need to infinitely move 
non deadlocked processes.

Let $\deadOne{x}$ be the set of deadlocked states in the run $x$ 
that are only deadlocked if there is no other process in the same state, 
and let $\mD_1$ be the set of processes deadlocked 
in the run $x$ in $\deadOne{x}$. 
Let $\deadTwo{x}$ be the set of states that are deadlocked 
in the run $x$ even if there is another process in the same state. 

We note the following:
\li
\- $|\mD_1| = |\deadOne{x}| \leq |B|$;

\- $\deadOne{x} \cap \deadTwo{x} = \emptyset$;

\- $\visFin{\mB\smi \mD_1}{x} \cap \deadOne{x} \neq \emptyset$
   is possible, because a state from $\visFin{\mB\smi \mD_1}{x}$ 
   can first be visited by a process in $\mB \smi \mD_1$, 
   and later be deadlocked because of the process in $\mD_1$;

\- $\deadTwo{x} \subseteq \visInf{\mB\smi \mD^1}{x}$,
   and hence $\visFin{\mB \smi \mD_1}{x} \cap \deadTwo{x} = \emptyset$.
\il

The construction has two phases. 
The first phase is as follows.
\li
\-[a.] For every $p \in \{A\} \cup \mD_1$, set $y(p)=x(p)$;

\-[b.] for every $q \in \deadTwo{x}$, 
       devote one process of $\cutoffsys$ that floods it;

\-[c.] for every $q \in \visInf{\mB \smi \mD_1}{x} \smi \deadTwo{x}$,
       devote two processes of $\cutoffsys$ that flood it;

\-[d.] for every $q \in \visFin{\mB \smi \mD_1}{x}$, 
       devote one process of $\cutoffsys$ that floods it
       and then evacuates into $\visInf{\mB \smi \mD_1}{x}$;
       and

\-[e.] let other processes (if any) mimic some process from (c).
\il
After this phase all $B$ processes will be in 
$\visInf{\mB \smi \mD_1}{x} \cup \deadOne{x}$. 

The second phase applies to processes in 
$\visInf{\mB \smi \mD_1}{x} \smi \deadTwo{x}$ the fair extension%
\footnote{The fair extension requires the run $x$ to be unconditionally-fair, 
          but here we have a run in which all processes that are not deadlocked
          move infinitely often.
          To adapt the construction to this case:
          copy local runs of processes $\{A\} \cup \mD_1$,
          and do not extend local runs of processes that are in a
          state in $\dead_2$.}.

% The resulting configuration sequence is a run of $\cutoffsys$ by correctness of the flooding, evacuation, fair extension, interleaving and destuttering constructions\ak{prove}. Furthermore, for every $q \in \deadOne$ we have exactly one process in $\cutoffsys$ that eventually stays in $q$, and for every $q \in \deadTwo$ -- at least one such process; they are eventually deadlocked in $q$ because the states that appear infinitely often in the run $x$ of $\cutoffsys$ are the same as in the resulting run of $\largesys$.

How many processes does the construction use? 
Note that the sets 
$\deadOne{x} \cup \visFin{\mB \smi \mD_1}{x}$, 
$\deadTwo{x}$, 
$\visInf{\mB \smi \mD_1}{x} \smi \deadTwo{x}$ 
are disjoint, thus:
\begin{align}
& | \visFin{\mB \smi \mD_1}{x} | + |\deadOne{x}| + |\deadTwo{x}| + 2| \visInf{\mB \smi \mD_1}{x} \smi \deadTwo{x} | \leq \label{disj:eq:1} \\
% 
& 2|\visFin{\mB \smi \mD_1}{x} \cup \deadOne{x}| + |\deadTwo{x}| + 2| \visInf{\mB \smi \mD_1}{x} \smi \deadTwo{x} | \leq \label{disj:eq:2} \\
% 
& |B| + |\visFin{\mB \smi \mD_1}{x} \cup \deadOne{x}| + | \visInf{\mB \smi \mD_1}{x} \smi \deadTwo{x} | \leq 2|B|  \nonumber
\end{align}
Let us reduce the estimate to $\leq 2|B|-1$:
\li
  \- assume that $\deadTwo{x} = \emptyset$ 
     (otherwise, Eq.\ref{disj:eq:1} and the sets disjointness give $2|B|-1$);
     and

  \- assume that $ \visFin{\mB \smi \mD_1}{x} \neq \emptyset$ 
     (the other case together with eq.\ref{disj:eq:2}, 
      the sets disjointness, and the first item gives $2|B|-1$);

  \- hence, the construction in step (d) evacuates the process in 
     $q \in \visFin{\mB \smi \mD_1}{x}$ 
     into 
     $ \visInf{\mB \smi \mD_1}{x} \smi \deadTwo{x}$. 
     Hence modify step (c) of the construction 
     and for $q$ devote a single process of $\cutoffsys$ that floods it. 
     This will give $\leq 2|B|-1$.
\il
This concludes the proof.

\end{proof}



\begin{tightness}[Disj, Deadlocks, Fair]
\label{obs:disj:fair_tight_deadlock}
The cutoff $c=2|B|-1$ for deadlock detection in disjunctive systems on strong-fair or finite runs is tight.
I.e., for any $k$ there are templates $(A,B)$ with $|B|=k$ such that:
$$
(A,B)^{(1,2|B|-2)} \textit{ does not have a deadlock, but } (A,B)^{(1,2|B|-1)} \textit { does}.
$$
\end{tightness}
\begin{proof}
Figure~\ref{gua:fig:tight_disj_dead_fair}
shows process templates $(A,B)$ such that any system $\largesys$ with $n\leq 2|B|-2$ does not deadlock on strong-fair runs, but larger systems do.
% 
\begin{figure}[htpb]
\centering
\begin{subfigure}[b]{0.45\textwidth}\center
\scalebox{0.75}{% !TEX root = table.tex
\begin{tikzpicture}[node distance=2.5cm,inner sep=1pt,minimum size=0.5mm,->,>=latex]
% \tikzset{every state/.style={circle,minimum size=.5cm,inner sep=.03cm}}


\node[initial left, state] (a_1) {};
% \node[state] (a_2) [right of=a_1] {$a_2$};
\node (dots) [right of=a_1] {$\ldots$};
\node[state] (a_k) [right of=dots] {}; 
\node[state] (r) [below=0.8cm of dots] {$r_A$};

\path (a_1) edge [above] node {$\disj{1_B}$} (dots);
\path (dots) edge [above] node {$\disj{{k-1}_B}$} (a_k);
\path (a_1) edge [bend right=15] node {} (r);
\path (a_k) edge [bend left=15] node {} (r);
\path (dots) edge [above, dotted] node {} (r);
\path (a_k) edge [bend right=35,above] node {$\disj{k_B}$} (a_1);

% \path (a_1) [loop above] edge [right] node {$\disj{a_1}$} (a_1);
% \path (a_2) [loop above] edge [right] node {$\disj{a_2}$} (a_2);
% \path (a_k) [loop above] edge [right] node {$\disj{a_k}$} (a_k);
\path (r) [loop below] edge [right] node {} (r);



\end{tikzpicture}

}
\label{fig:disj:tight_fair_deadlock_tmplA}
\caption*{Template A}
\end{subfigure}
\begin{subfigure}[b]{0.45\textwidth}\center
\scalebox{0.75}{% !TEX root = table.tex
\begin{tikzpicture}[node distance=2.5cm,inner sep=1pt,minimum size=0.5mm,->,>=latex]
% \tikzset{every state/.style={circle,minimum size=.5cm,inner sep=.03cm}}


\node[state,initial left] (b_1) {$1_B$};
\node (dots) [right of=b_1] {$\ldots$};
\node[state] (b_k) [right of=dots] {$k_B$}; 
\node[state] (r) [below=0.8cm of dots] {};

\path (b_1) edge [above] node {$\disj{1_B}$} (dots);
\path (dots) edge [above] node {$\disj{{k-1}_B}$} (b_k);
\path (b_1) edge [bend right=15] node {} (r);
\path (b_k) edge [bend left=15] node {} (r);
\path (dots) edge [above, dotted] node {} (r);

\path (b_1) [loop above] edge [right] node {$\disj{1_B}$} (b_1);
\path (b_k) [loop above] edge [right] node {$\disj{k_B}$} (b_k);
\path (r) [loop below] edge [right] node {$\disj{r_A}$} (r);
\path (dots) [loop above,dotted,in=60,out=120,distance=12mm] edge [right] node {} (dots);



\end{tikzpicture}

}
\caption*{Template B}
\end{subfigure}
\caption{Templates $(A,B)$ used in Tightness~\ref{obs:disj:fair_tight_deadlock}.}
\label{gua:fig:tight_disj_dead_fair}
\end{figure}
% 
\end{proof}
