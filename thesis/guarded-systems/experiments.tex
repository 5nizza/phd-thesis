\section{Experiments} \label{gua:sec:experiments}

(The experiments were done by S. Au{\ss}erlechner,
 as a part of his Master Thesis.)

We used our new cutoff results in the context of parameterized synthesis to 
automatically construct process templates with safety and liveness guarantees 
for guarded systems with an arbitrary number of components.
Our prototype is an extension of the parameterized synthesis tool 
{\sc Party}~\cite{party}.
It synthesizes process templates based on the semi-decision procedure 
described in Section~\ref{gua:sec:paramsynt}. 
Bounded template synthesis is 
implemented as an extension of the bounded synthesis approach~\cite{BS}, with
LTL3BA~\cite{LTL3BA} for translation of specifications to automata, and
SMT solver Z3~\cite{Moura08} as backend. 

%In the cutoff detection step, we apply the cutoff results from 
%Section~\ref{gua:sec:cutoffs} modularly, i.e., we calculate cutoffs and convert 
%to universal co-Büchi tree automaton (UCT)\sj{remind/use long version} for every conjunct of the specification separately (an optimization 
%mentioned in~\cite{KhalimovJB13b}). Finally, we produce the SMT encoding, 
%conjoining the SMT constraints for all parts. 
%
As a proof of concept, we synthesized implementations for two small 
examples (on a single core of a 3.5GHz Intel i7 CPU with 4GB RAM):
\begin{itemize}
\item a closed disjunctive system $\largesys$ where $A$ controls an output 
signal $w$, and every $B_i$ a signal $g_i$, with fairness assumption and specification
$\GF(\neg w) \wedge \GF w \wedge 
\bigwedge_{i} \left[\GF(w \wedge g_i) \wedge  \GF g_i \wedge \GF(\neg g_i) 
\right]$,
i.e., all processes must toggle their signals 
infinitely often, and every process $B^i$ must ``meet'' infinitely often with 
$A$ when both signals are enabled. The implementation below has been synthesized with a cutoff of $2\card{B}+1-1=4$ for the property and $2\card{B}-1=3$ for the deadlock detection within one minute.

\item an open conjunctive system $(B)^{(n)}$ (i.e., $A$ can be arbitrary and $B$ does not react to it), where each process $B_i$ gets requests of high and low priority as input ($rh_i, rl_i$), and controls ouputs that represent grants to these requests ($gh_i, gl_i$). The specification requires i) `every request $rh$ should eventually be granted ($gh_i$)', ii) `every request $rl_i$ should eventually be granted ($gl_i$) unless there is a simultaneous request $rh_i$', iii) `there should be no spurious grants', iv) `mutual exclusion of grants (locally and globally)'. The implementation above has been synthesized with a cutoff of $3$ for properties, and $4$ for deadlock detection within about $45$ minutes. Note that we do not restrict the implementation to $1$-conjunctive systems, and therefore the cutoff does not guarantee absence of local deadlocks in systems of arbitrary size.
\end{itemize}

\begin{figure}[ptb]
\centering
\begin{subfigure}[b]{0.45\textwidth}\center
\scalebox{0.7}{\begin{tikzpicture}[->,>=stealth',shorten >=1pt,auto,node distance=1.6cm,
                    semithick]
  \tikzstyle{every state}=[align=center,anchor=center]
  \tikzstyle{every edge} = [align=center,draw=black]
  \tikzstyle{boxLabel} = [yshift=-0.4cm]
  \tikzstyle{box} = [draw=black, inner sep=0.75cm]
	
  \node[state] (t00) [label={left:$\neg w$}]                    {$0_A$};
  \node[state] (t01) [below of=t00,label={left:$w$}] {$1_A$};
  \coordinate (t00') at ($(t00.west) + (-1.0cm,0.5cm)$);
  \coordinate (t01') at ($(t01.east) + (+0.75cm,-0.25cm)$);
  \node (U1) [draw=black, fit=(t00') (t01'), inner sep=0.75cm] {} ;
  \node [boxLabel] at (U1.north) {Template $A$};
  
  \path (t00) edge [bend right,left] node {$\exists \left\{0_B, 1_B\right\}$} (t01)
            (t01) edge [bend right, right] node {$\exists \left\{1_B\right\}$} (t00);

   
   
  \node[state] (t10) [label={left:$\neg g$},right of=t00,node distance=4.5cm] {$0_B$};
  \node[state] (t11) [label={left:$g$}, below of=t10] {$1_B$};
  \coordinate (t10') at ($(t10.west) + (-0.75cm,0.5cm)$);
  \coordinate (t11') at ($(t11.east) + (+0.75cm,-0.25cm)$);
  \node (U2) [box,fit=(t10') (t11')] {} ;
  \node [boxLabel] at (U2.north) {Template $B$};
  
  \path (t10) edge [bend right,left] node {$\exists \left\{1_A\right\}$} (t11)
            (t11) edge [bend right, right] node {$\exists \left\{1_A\right\}$} (t10);
  %\draw[ltsBox,label={above:x},anchor=center] ($(t10.north west)+(-1.7,0.6)$)  rectangle ($(t11.south east)+(1.7,-0.6)$); 
\end{tikzpicture}
}
\caption{Disjunctive system}
\label{fig:disjunctiveLTS}
\end{subfigure}
\begin{subfigure}[b]{0.45\textwidth}\center
\scalebox{0.7}{\begin{tikzpicture}[->,>=stealth',shorten >=1pt,auto,node distance=1.5cm,
                    semithick]
  \tikzstyle{every state}=[align=center, anchor=center]
  \tikzstyle{every edge} = [align=center,draw=black,]

  \node[state] (A) [label={below:$$}] {$0_A$};
  \node[] (BC) [below of=A] {$$};
  \node[state] (B) [left of=BC, label={below:$gh$}] {$1_A$};
  \node[state] (C) [right of=BC, label={below:$gl$},] {$2_A$};


  \path 
  (A) edge [loop above] node {$\neg rh \wedge \neg rl : \forall \{0_A\}$} (A)
  (A) edge [left, bend right=10] node {$rh: \forall \{0_A\}$} (B)
  (B) edge [right, pos=0.3, bend right=10] node {$*: \forall \{0_A\}$} (A)
  (A) edge [right, bend left=10] node {$\neg rh \wedge rl: \forall \{0_A\}$} (C)
  (C) edge [right, bend left=10] node {$$} (A);
\end{tikzpicture}}
\label{fig:conjunctiveLTS3}
% \caption{Templates $(A,B)$ used to prove the tightness of the cutoffs for properties $\pexists h(A^1,B^1)$ (Observation~\ref{obs:disj:tight_prop})\ak{check me}.}
\label{fig:experiments}
\caption{Conjunctive system}
\end{subfigure}
\caption{Synthesized systems}
\end{figure}
