\section{Preliminaries} \label{gua:sec:prelim}\label{gua:sec:definitions}
Many definitions intersect with those defined in previous chapters,
but to keep the chapter self-contained we define them here.

Notation:
$\bbB = \{\true,\false\}$ is the set of Boolean values,
$\bbN$ is the set of natural numbers (excluding $0$),
$\bbN_0 = \bbN\cup\{0\}$,
$[k]$ is the set $\{i \in \bbN \| i \leq k\}$
and $[0..k]$ is the set $[k] \cup \{0\}$ for $k \in \bbN$.
For a sequence $x=x_1x_2\ldots$ denote the $i$-$j$-subsequence as $x\slice{i}{j}$,
i.e., ${x\slice{i}{j}}=x_i \ldots x_j$.


\subsection{System Model} \label{gua:sec:model}

We consider systems $A {\parallel} B^n$, usually written $\largesys$, 
consisting of
one copy of a process template $A$ and $n$ copies of a process template $B$,
in an interleaving parallel composition.%
%% AK: moved to a separate note, to be able to explain why we _cannot_ generalize for 1-conj
%\footnote{As shown in \cite{Emerson00}, cutoffs for this case generalize to cutoffs 
%          for systems of the form $A^m {\parallel} B^n$, and further 
%          to systems with an arbitrary number of process templates 
%          $U_1^{n_1} {\parallel} \ldots {\parallel} U_m^{n_m}$.} 
We distinguish objects that belong to different templates by indexing them with
the template. E.g., for process template $U \in \{A,B\}$, $Q_U$ is the set of
states of $U$. For this section, fix two disjoint finite sets $Q_A$, $Q_B$ as
sets of states of process templates $A$ and $B$, and a positive integer $n$.

\parbf{Processes} A \emph{process template} 
 is a transition system
  $U=(\stateset, \init, \inputs, \delta)$ with 
	\begin{itemize}
	\item $\stateset$ is a finite set of states including the
  initial state $\init$,
	\item $\inputs$ is a finite input alphabet,
	\item $\delta: \stateset \times \inputs \times \mP(Q_A \cupdot Q_B) \times \stateset$ is a guarded transition relation.
	\end{itemize}
A process template is \emph{closed} if $\inputs = \emptyset$, and otherwise \emph{open}.

By $\transition{q_i}{q_j}{e:g}$
we denote a process transition from $q_i$ to $q_j$
for input $e \in \Sigma$ and guarded by guard $g \in \mP(Q_A \cupdot Q_B)$.
We skip the input $e$ and guard $g$
if they are not important or can be inferred from the context.

We define the size $\card{U}$ of a process template $U \in \{A,B\}$ as $\card{\stateset_U}$. A copy of a template $U$ will be called a \emph{$U$-process}.
Different $B$-processes are distinguished by subscript, i.e., for $i \in [1..n]$, $B_i$ is the $i$th copy of $B$, and $\state_{B_i}$ is a state of $B_i$. A state of the $A$-process is denoted by $q_A$. 

For the rest of this subsection, fix templates $A$ and $B$. We assume that $\inputs_A \cap \inputs_B = \emptyset$. We will also write $p$ for a process in $\{ A, B_1, \ldots, B_n\}$, unless $p$ is specified explicitly.
We often denote the set $\{B_1,...,B_n\}$ as $\mB$.


\parbf{Disjunctive and conjunctive systems}
In a system $\largesys$,
consider the global state $s = (\state_A,\state_{B_1},\ldots,\state_{B_n})$ and
global input $e=(\localin_A,\localin_{B_1},\ldots,\localin_{B_n})$.
We write $s(p)$ for $q_p$, and $e(p)$ for $\sigma_p$.
A local transition $(\state_p,\localin_p,g,\state_p') \in \delta_U$ of a process $p$ is \emph{enabled for $s$ and $e$}
if the \emph{guard} $g$ is satisfied by the state $s$ wrt.\ the process $p$, written $(s,p) \models g$ (defined below).
The semantics of $(s,p) \models g$ differs for disjunctive and conjunctive systems:
%
\begin{align*}
\text{In disjunctive systems: } & (s,p) \models g \text{~~~iff~~~} 
\exists p' \in \{A,B_1,\ldots,B_n\} \setminus \{p\}:\ \ \state_{p'} \in g. \\
\text{In conjunctive systems: } & (s,p) \models g \text{~~~iff~~~} 
\forall p' \in \{A,B_1,\ldots,B_n\} \setminus \{p\}:\ \ \state_{p'} \in g.
\end{align*}

Note that we check containment in the guard (disjunctively or conjunctively) 
only for local states of processes \emph{different from} $p$. A process is \emph{enabled} for $s$ and $e$ if at least one of its transitions is enabled for $s$ and $e$, otherwise it is \emph{disabled}.

Like Emerson and Kahlon~\cite{Emerson00}, 
we assume that in conjunctive systems $\init_A$ and $\init_B$ are contained in all guards,
i.e., they act as neutral states.
Furthermore, we call a conjunctive system \emph{$1$-conjunctive} if every guard is of the form $(Q_A \cupdot Q_B) \setminus \{q\}$ for some $q \in Q_A\cupdot Q_B$.

Then, \largesys is defined as the transition 
system $(S,\init_S,\globIn,\delta)$ with 
\begin{itemize}
\item set of global states $S = \stateset_A \times \stateset_B^{n}$, 
\item global initial state $\init_S = (\initstate_A,\initstate_B,\ldots,\initstate_B)$, 
\item set of global inputs $\globIn = (\inputs_A) \times (\inputs_B)^{n}$,
\item and global transition relation $\delta \subseteq S \times \globIn \times S$ with $(s,e,s') \in \delta$ iff 
\begin{enumerate}[label=\roman*)] 
  \item $s=(\state_A,\state_{B_1},\ldots,\state_{B_n})$, 
  \item $e=(\localin_A, \localin_{B_1},\ldots,\localin_{B_n})$, and 
  \item $s'$ is obtained from $s$ by replacing one local state $\state_p$ with a new local state $\state_p'$, where $p$ is a $U$-process with local transition $(\state_{p},\localin_{p},g,\state_p') \in \delta_U$ and $(s,p) \models g$. 
        Thus, we consider so-called interleaved systems,
        where in each step exactly one process transits.
\end{enumerate}
\end{itemize}
We say that a system $\largesys$ is \emph{of type} $(A,B)$. It is called a
\emph{conjunctive system} if guards are interpreted conjunctively, and a
\emph{disjunctive system} if guards are interpreted disjunctively. 
A system is \emph{closed} if all of its templates are closed.


\parbf{Runs} 
A \emph{configuration} of a system is a triple $(s,e,p)$, where $s \in S$, $e 
\in \globIn$, and $p$ is either a system process, or the special symbol $\bot$.
 A \emph{path} of a system is a configuration sequence 
$x = (s_1,e_1,p_1),(s_2,e_2,p_2),\ldots$ such that, for all $\time < |x|$, there is a 
transition $(s_\time,e_\time,s_{\time+1}) \in \delta$ based on a local 
transition of process $p_\time$. We say that process 
$p_\time$ \emph{moves} at \emph{moment} $\time$. 
Configuration $(s,e,\bot)$ appears
 iff all processes are disabled for $s$ and $e$.
Also, for every $p$ and $\time < |x|$: 
either $e_{\time+1}(p) = e_\time(p)$ or process $p$ moves at moment $\time$. 
That is, the environment keeps the input to each process unchanged until 
the process can read it.\footnote{By only considering inputs that are actually processed, we 
approximate an 
action-based semantics. Paths that do not fulfill this requirement are not 
very interesting, since the environment can violate any interesting 
specification that involves input signals by manipulating them when the 
corresponding process is not allowed to move.} 

A system \emph{run} is a maximal path starting in the initial state. Runs are either infinite, or they end in a configuration $(s,e,\bot)$. We say that a run is \emph{initializing} if every 
%$B$-process 
process
that moves infinitely often also visits 
%$\initstate_B$ 
its $\initstate$ 
infinitely often.

Given a system path $x = (s_1,e_1,p_1),(s_2,e_2,p_2),\ldots$ and a process $p$, the \emph{local path} of $p$ in $x$ is the projection $x(p) = (s_1(p),e_1(p)),(s_2(p),e_2(p)),\ldots$ of $x$ onto local states and inputs of $p$.
Similarly, we define the projection on two processes $p_1,p_2$ denoted by $x(p_1,p_2)$.

%The \emph{destuttering} $\destutter(x)$\ak{make it work with inf runs} of a (local) path \sj{local path not defined} $x=x_0,x_1,\ldots$ is obtained by removing stuttering steps from the sequence, i.e., $\destutter(x)$ is the maximal subsequence $x'$ of $x$ such that for every $\time$ we have $x'_\time \neq x'_{\time+1}$. Two (local) paths $x$ and $y$ are \emph{stutter-equivalent}, written $x \simeq y$, if $\destutter(x)=\destutter(y)$. Define an extension of $\destutter$ to sets of paths in the obvious way. Then two systems $S_1, S_2$ are \emph{stutter-equivalent}, written $S_1 \simeq S_2$, if $\destutter(X_1) = \destutter(X_2)$, where $X_i$ is the set of all infinite runs of system $S_i$.

\parbf{Deadlocks and fairness}
A run is \emph{globally deadlocked} if it is finite.
An infinite run is \emph{locally deadlocked} for process $p$ if there exists $\time$ such that $p$ is disabled for all $s_{\time'},e_{\time'}$ with $\time'\ge \time$. A run is \emph{deadlocked} if it is locally or globally deadlocked.
A system \emph{has a (local/global) deadlock} if it has a (locally/globally) deadlocked run. Note that the absence of local deadlocks for all $p$ implies the absence of global deadlocks, but not the other way around.

A run $(s_1,e_1,p_1), (s_2,e_2,p_2),...$ is \emph{unconditionally-fair} if every process moves infinitely often. 
A run is \emph{strong-fair} if it is infinite and, for every process $p$, if $p$ is enabled infinitely often, then $p$ moves infinitely often.
%\sj{weak fairness needed?} Finally, $x$ is \emph{weak-fair} if it is infinite and for every process $p$, if there exists $t$ such that $p$ is enabled for every $s_{\time'}, e_{\time'}$ with $\time' \ge \time$, then $p$ moves infinitely often.
We will discuss the role of deadlocks and fairness in synthesis in Section~\ref{gua:sec:paramsynt}.

\begin{remark}[$A^m {\parallel} B^n$]
One usually starts with studying parameterized systems of the form $A^n$ (having one process template),
then proceeds to systems of the form $A^m {\parallel} B^n$ (having two templates)
and $U_1^{n_1} {\parallel} \ldots {\parallel} U_m^{n_m}$ (having an arbitrary fixed number of templates).
Our work studies systems $A {\parallel} B^n$,
which have one $A$-process and a parameterized number of $B$-processes,
because the results for such systems can be generalized to systems $U_1^{n_1} {\parallel} \ldots {\parallel} U_m^{n_m}$
(see~\cite{Emerson00} for details).
This generalization works for our results as well,
except for the cutoffs for deadlock detection that are restricted to 1-conjunctive systems of the form $A\parallel B^n$
(Section~\ref{gua:sec:cutoffs}).
\end{remark}

\subsection{Specifications}
\label{gua:sec:semantics}
Fix templates $(A,B)$.
We consider formulas in $\LTLmX$---$\LTL$ without the next-time operator $\nextt$---%
that are prefixed by path quantifiers $\E$ or $\A$
(for LTL and path quantifiers see Section~\ref{defs:ctlstar}).
Let $h(A,B_{i_1},\ldots,B_{i_k})$ be an $\LTLmX$ formula over atomic propositions from $Q_A \cup \Sigma_A$ and indexed propositions from $(Q_B \cup \Sigma_B) \times \{i_1,\ldots,i_k\}$.
For a system $\largesys$ with $n \geq k$ and every $i_j \in [1..n]$,
satisfaction of $\A h(A,B_{i_1},\ldots,B_{i_k})$ and $\E h(A,B_{i_1},\ldots,B_{i_k})$ is defined in the usual way.


\parbf{Parameterized specifications} \label{gua:sec:parameterized}
A \emph{parameterized specification} is a temporal logic formula
with indexed atomic propositions and quantification over indices. 
We consider formulas of the forms
$\forall{i_1,\ldots,i_k.} \A h(A,B_{i_1},\ldots,B_{i_k})$ and\\ 
$\forall{i_1,\ldots,i_k.} \E h(A,B_{i_1},\ldots,B_{i_k})$. 
For a given $n \geq k$, 
$$
\largesys \models \forall{i_1,{\ldots},i_k.} \A h(A,B_{i_1},{\ldots},B_{i_k})
$$
~iff~
$$
\largesys \models \!\!\!\!\!\!\!\!\bigwedge_{j_1 \neq {\ldots} \neq j_k \in [1..n]}\!\!\!\!\!\!\!\!\A h(A,B_{j_1},{\ldots},B_{j_k}).
$$ 
By symmetry of guarded systems (see~\cite{Emerson00}),
the second formula is equivalent to
$\largesys \models \A h(A,B_1,\ldots,B_k)$. 
The formula $\A h(A,B_1,\ldots,B_k)$ is denoted by $\A h(A,B^{(k)})$, 
and we often use it instead of the original $\forall{i_1,\ldots,i_k.} \A h(A,B_{i_1},...,B_{i_k})$.
For formulas with the path quantifier $\E$,
satisfaction is defined analogously
and is equivalent to satisfaction of $\E h(A,B^{(k)})$.
\begin{example}
Consider the formula
$$
\forall{i_1,i_2}.\A \big(\G (r_{i_1} \impl \F g_{i_1}) \land \G \neg (g_{i_1} \land g_{i_2})\big).
$$
By our definition, its satisfaction by a system $(A,B)^{(1,3)}$ means
\begin{align*}
(A,B)^{(1,3)} \models
\A \left(
\begin{aligned}
&\G(r_1 \impl \F g_1) \land \G(r_2 \impl \F g_2) \land \G(r_3 \impl \F g_3) \land \\
&\G \neg (g_1 \land g_2) \land \G \neg (g_1 \land g_3) \land \G \neg (g_2 \land g_3)
\end{aligned}
\right),
\end{align*}
where $g_1$ and $r_1$ refer to the propositions $g$ and $r$ of the process $B_1$,
$g_2$ and $r_2$ belong to $B_2$, and so on.
By symmetry, the latter satisfaction is equivalent to
\begin{align*}
(A,B)^{(1,3)} \models
\A \left(
\begin{aligned}
&\G(r_1 \impl \F g_1) \land \G(r_2 \impl \F g_2) \land \\
&\G \neg (g_1 \land g_2)
\end{aligned}
\right).
\end{align*}
Note that this formula talks about processes $B_1$ and $B_2$, but does not mention $B_3$.
\end{example}


\parbf{Specification of fairness and local deadlocks}
It is often convenient to express fairness assumptions and local deadlocks 
as parameterized specifications.
To this end,
define auxiliary atomic propositions $\sched_p$ and $\enabled_p$ for every process $p$ of system $(A,B)^{(1,n)}$. At moment $\time$ of a given run $(s_1,e_1,p_1),(s_2,e_2,p_2), \ldots$, let $\sched_p$ be true whenever $p_\time = p$, and let $\enabled_p$ be true if $p$ is enabled for $s_\time, e_\time$. Note that we only allow the use of these propositions to define fairness, but not in general specifications.
Then, an infinite run is 
\begin{itemize}
\item \emph{local-deadlock-free} if it satisfies $\forall{p}. \GF \enabled_p$, abbreviated as $\spec_{\neg dead}$,
\item \emph{strong-fair} if it satisfies $\forall{p}. \GF \enabled_p \impl \GF \sched_p$, abbreviated as $\spec_{strong}$, and 
\item \emph{unconditionally-fair} if it satisfies $\forall{p}. \GF \sched_p$, abbreviated as $\spec_{uncond}$.
%\item \sj{needed?:}\emph{weak-fair} if it satisfies $\forall{p}. \A \spec_{weak}$, where $\spec_{weak} = \FG \enabled_p \impl \GF \sched_p$.
\end{itemize}

If $f \in \{strong, uncond\}$ is a fairness notion and 
$\A h(A,B^{(k)})$
a specification, then we write
$\A_{f} h(A,B^{(k)})$ for $\A (\spec_{f} 
\rightarrow h(A,B^{(k)}))$.
Similarly, we write $\E_{f} h(A,B^{(k)})$ for $\E (\spec_{f} \land h(A,B^{(k)}))$.



\subsection{Model Checking and Synthesis Problems}
\label{gua:sec:nonparameterized_synthesis}
%
Given a system $\largesys$ and a specification $\A h(A,B^{(k)})$, where $n \ge k$. Then:
\begin{itemize}
\item the \emph{model checking problem} is to decide whether $\largesys \models \A h(A,B^{(k)})$,
\item the \emph{deadlock detection problem} is to decide whether $\largesys$
      does not have global nor local deadlocks,
%\item the \emph{deadlock detection problem} is to decide whether all runs of $\largesys$
%are infinite and $\largesys \models \A \spec_{\neg dead}$, 
%i.e., there are no local deadlocks,
\item the \emph{parameterized model checking problem} (PMCP) is to decide whether $\forall m \ge n:\ (A,B)^{(1,m)} \models \A h(A,B^{(k)})$, and 
%\item the \emph{parameterized deadlock detection problem} is to decide whether for all $m \ge n$, all runs of $(A,B)^{(1,m)}$ are infinite and $(A,B)^{(1,m)} \models \A \spec_{\neg dead}$.
\item the \emph{parameterized deadlock detection problem} is to decide whether, 
      for all $m \ge n$, $(A,B)^{(1,m)}$ does not have global nor local deadlocks.
\end{itemize}
For a given number $n \in \bbN$ and specification $\A h(A,B^{(k)})$ with $n \ge k$,
\begin{itemize}
\item the \emph{template synthesis problem} is to find process templates $A,B$ such that
$\largesys \models \A h(A,B^{(k)})$ and $\largesys$ does not have global deadlocks\footnote{\label{footnote:local-deadlocks}Here we do not explicitly mention local deadlocks because they can be specified as a part of $\A h(A,B^{(k)})$.}
\item 
the \emph{bounded template synthesis problem} for a pair of bounds $(\bound_A,\bound_B) \in \bbN \times \bbN$ 
is to solve the template synthesis problem with
$\card{A} \leq \bound_A$ and $\card{B} \leq \bound_B$.
\item the \emph{parameterized template synthesis problem} is to find process templates $A,B$ such that $\forall m \ge n:\ (A,B)^{(1,m)} \models \A h(A,B^{(k)})$ and $(A,B)^{(1,m)}$ does not have global deadlocks\footnoteref{footnote:local-deadlocks}.
\end{itemize}
Similarly, we define problems for specifications having $\E$ instead of $\A$.
The definitions can be flavored with different notions of fairness.
