\section{Proof Structure}\label{gua:sec:proof-structure}

The proofs for the cutoff results, new and original,
are based on two lemmas, Monotonicity and Bounding~\cite{Emerson00}.
When combined together, the lemmas give a cutoff.
We state the lemmas, and discuss them in the context of deadlock detection and fairness.
The detailed proofs are in Sections~\ref{gua:sec:proofs-disj} and \ref{gua:sec:proofs-conj}.
Note that we only consider properties of the form $h(A,B^{(1)})$---the
proof ideas extend to general properties $h(A,B^{(k)})$ without difficulty.
Similarly, in most cases the proof ideas extend to open systems
without major difficulties---mainly because when we construct a simulating
run, we have the freedom to choose the input that is needed.
% AK: in the main text of the paper no difficulties are seen
Only for the case of deadlock detection we have to handle open systems explicitly.

\smallskip
\noindent
{\bf 1) Monotonicity lemma:} if a behavior is possible in a (conjunctive or disjunctive) system with $n \in \Nat$ copies of $B$,
then it is also possible in a (conjunctive or disjunctive resp.) system with one additional process:
\[
\largesys \models \pexists h(A,B^{(1)}) 
~\implies~
(A,B)^{(1, n+1)} \models \pexists h(A,B^{(1)}), 
\]
and if a deadlock is possible in $(A,B)^{(1, n)}$, then it is possible in $(A,B)^{(1, n+1)}$.
%

\parit{Discussion}
The lemma is easy to prove for properties 
$\pexists h(A,B^{(1)})$ in both disjunctive and conjunctive systems, by letting the 
additional process stay in its initial state $\init_B$ forever (see~\cite{Emerson00}).
This cannot disable transitions with disjunctive guards, as
these check for \emph{existence} of a local state in another process (and we 
do not remove any processes), and it cannot disable conjunctive guards since 
they contain $\init_B$ by assumption. 
However, this construction violates fairness, since the new process
never moves. This can be resolved in the disjunctive case by letting the
additional process mimic all transitions of an existing process. But in
general this does not work in conjunctive systems (due to the non-reflexive
interpretation of guards).
For this case and for deadlock detection, the proof is not
trivial and may only work for $n \geq c$, for some lower bound $c \in \Nat$.
The following sections provide the details.


\smallskip
\noindent
{\bf 2) Bounding lemma:} there exists a number $c \in \Nat$ such that
a behavior is possible in a system with $c$ copies of $B$ if it is possible in a system with
$n \geq c$ copies of process $B$:
\[
(A,B)^{(1, c)} \models \pexists h(A,B^{(1)})
~\impliedby~
(A,B)^{(1, n)} \models \pexists h(A,B^{(1)})
,
\]
and a deadlock is possible in \cutoffsys if it is possible in \largesys.

\parit{Discussion}
For disjunctive systems,
the main difficulty is that removing processes
might falsify guards of the local transitions of $A$ or $B_1$ in a given run.
To address this,
Emerson and Kahlon~\cite{Emerson00} came up with so-called flooding construction (described later).
For conjunctive systems,
removing processes from a run is easy for the case of infinite runs,
since a transition that was enabled before cannot become disabled.
Here, the difficulty is in preserving deadlocks,
because removing processes may enable processes that were deadlocked before.
The next sections explain how to address this.

\parbf{Tightness}
Recall from Section~\ref{gua:sec:paramsynt} that
$c$ is a tight cutoff iff $c$ is a cutoff and there are templates $(A,B)$ and a property $\Phi$,
such that
$$\sys{c-1} \not\models \Phi \textit{ and } \sys{c} \models \Phi.$$
For deadlock detection this is equivalent to:
$\sys{c-1}$ does not have a deadlock but $\sys{c}$ does.
%For properties of the form $\E h(A,B^{(1)})$ and $\A h(A,B^{(1)})$,
%due to the equivalence ``$\sys(n) \models \E h$'' $\equiv$ ``$\sys{n} \not\models \A\neg h$'',
%this is equivalent to having:
%$\sys{c-1} \not\models \E h(A,B^{(1)})$ and $\sys{c} \models \E h(A,B^{(1)})$.
To prove tightness, we provide a template $(A,B)$ and a property.

The next sections contains all the proofs of the results in the table.
For each row and column, we prove monotonicity and bounding lemmas,
as well as tightness.
Note that for simplicity the proofs are for the case of $h(A,B_1)$,
while the generalization to the case $h(A,B^{(k)})$ follows.
