\chapter{Introduction}

%\parbf{What is reactive synthesis?}
{\small
\noindent
\li
\-[Dave:~] Hey, Elli, how can I calculate the week number from the date?
\vspace{-3mm}
\-[Elli:~] ...prints the C-function.
\vspace{-3mm}
\-[Dave:~] Great. Can this function also output three last requested dates?
\vspace{-3mm}
\-[Elli:~] ...prints another C-function.
\vspace{-3mm}
\-[Dave:~] Thanks. Can you also make it output the name of the requesting person?
\vspace{-3mm}
\-[Elli:~] I am sorry, Dave, I am afraid I can't do that.
\il
}

In synthesis,
we describe the required behaviour and ask the computer to find the solution
with such a behaviour.
(In the dialog above,
 Dave asks Elli to find a function that,
 given a date, outputs the week number to which the date belongs.)
In \emph{reactive} synthesis, we are interested not in simple ``do-and-forget'' functions,
but rather in functions that interact with the user
akin to functions with an internal state.
(In the dialog above, the second C-function is reactive.)
It is not always possible to find a solution,
in which case the synthesizer (Elli) outputs ``specification is unrealizable''.
(In the last dialogue request, the specification became unrealizable,
 because the person name is not available to the function to be synthesized.)
%To summarise, reactive synthesis is an automatic way
%to translate a human intention expressed in some language
%into a system of some kind.

%\parbf{What is reactive synthesis problem?}

In 1963, Alonzo Church introduced the reactive synthesis problem~\cite{Church63}:
given a formula in Monadic Second Order Logic of One Successor,
and the inputs and the outputs of a circuit,
find such a circuit such that all behaviors of the circuit satisfy the formula.
(The circuit behaviour is an infinite string of inputs combined with the outputs.)
Church's problem was solved by Rabin~\cite{Rabin69} and
by B\"uchi and Landweber~\cite{BL69} in 1969.\ak{how?}

%\parbf{Recent progress of reactive synthesis}

Recent research in reactive synthesis focused on specifications
given in Linear Temporal Logic (LTL),
introduced by Pnueli~\cite{pnueli1977temporal} in 1977.
LTL has temporal operators, like $\G$ (always) and $\F$ (eventually),
and allows one to state properties like ``every request is eventually granted'':
$\G(r \impl \F g)$.
A system satisfies a given LTL property if all its computations satisfy it.
Pnueli and Rosner proved~\cite{DBLP:conf/popl/PnueliR89}
that the LTL synthesis problem is 2EXPTIME-complete.
Their approach translates a given LTL formula into a nondeterministic B\"uchi automaton,
then determinises it into a deterministic parity automaton
with the aid of involved Safra construction~\cite{Safra},
turns the automaton into a game, and solves the game.
Recent research focused on how to overcome the high complexity and Safra construction:
the work~\cite{Bloem12} considered the synthesis for a subset of LTL called GR(1),
the work~\cite{BS,KupfermanV05} considered bounding the system size and gave a name to Bounded Synthesis,
by combining the previous bounding with efficient data structures---Anti-chains Synthesis~\cite{Filiot11}.
The SYNTCOMP competition~\cite{syntcomp} is another recent initiative
with the goal to advance efficient synthesisers and popularise reactive synthesis.

%\parbf{The issues with reactive synthesis}

Despite substantial progress,
reactive synthesis is not as widespread as model checking.
The major reason, I believe,
is that writing the specifications---especially \emph{complete} specifications---is hard.
The issue is less pronounced in model checking,
because we do not need all the properties,
only those to model check.

%\parbf{In light of this issue, two directions to proceed}

In light of this issue, there are two directions to proceed.
First, we can develop synthesis approaches for richer logics,
which can ease writing the specifications.
Second, we can find application contexts where high specification costs are acceptable.
This thesis targets both directions:
we develop new synthesis approaches for the logic called \CTLstar,
and we delve into synthesis of distributed algorithms.

\subsection*{Part I: Excursion into Branching Logic}

%\parbf{when was CTL introduced?}

Computation Tree Logic (CTL)~\cite{ctl-origin}
was introduced by Emerson and Clarke in 1981
to circumvent the high complexity (PSPACE-complete) of the LTL model checking problem
and to be able to specify structural properties.
In 1986 Emerson and Halpern introduced a generalization,
Computation Tree Star Logic (\CTLstar)~\cite{ctlstar-origin},
that subsumes both CTL and LTL.

%\parbf{what is \CTLstar?}

In contrast to LTL, which reasons about (linear) computation runs,
\CTLstar reasons about (branching) computation trees.
We can get such a tree by unfolding the system transition structure.
\CTLstar has---in addition to temporal operators---path quantifiers:
$\A$ (on all paths) and $\E$ (there exists a path).
Such path quantifiers allow us to reason about branching structure of trees,
not just about their ``linear'' paths.
For example, \CTLstar formula ``$\AGEF reset$'' says:
``on all tree paths, from every tree node,
  there should be a path into a node where `reset' holds''.
We cannot express such a property using LTL alone.

%\parbf{what advantages does CTL* have over LTL?}

Despite \CTLstar being more expressible than LTL,
the complexity of \CTLstar synthesis (2EXPTIME-complete)
stays the same.
This prompted us to look into approaches to \CTLstar synthesis.

%\parbf{what is the standard solution to CTL*?}

The standard solution~\cite{informatio} to \CTLstar synthesis
turns the \CTLstar formula into an alternating hesitant tree automaton,
removes nondeterminism and derives a universal co-B\"uchi tree automaton,
determinises it using Safra construction~\cite{Safra} into a parity tree automaton,
and, finally, checks its non-emptiness.
If it is empty, then the specification is unrealisable,
otherwise we can extract the system from the proof of the non-emptiness.
This approach is hard to implement correctly and efficiently,
due to the involved Safra construction%
\footnote{It was a common belief that the Safra construction
  is difficult to implement and results in impractical algorithms.
  However, the belief might be wrong,
  as SYNTCOMP~\cite{syntcomp} in 2017 showed:
  the LTL synthesiser {\tt ltl-synt} that used Safra construction performed very well.}.

%\parbf{what is our contribution to CTL* synthesis?}

Part I contribution is two practical approaches to \CTLstar synthesis.

\subsubsection*{Contribution I.1: \CTLstar Bounded Synthesis}

We developed two bounded synthesis approaches for the \CTLstar specifications.
Let us recall how the SMT-based bounded synthesis by Schewe and Finkbeiner~\cite{BS} works:
we bound the system size, and
encode the resulting synthesis problem into an SMT query\footnote{%
  Satisfiability Modulo Theory (SMT)~\cite{SMT} query is a
  set of constraints over in a given theory.
  For example, in Linear Integer Arithmetic theory,
  the constraints talk about integer variables,
  use operations plus, minus, and the comparison relations.
  Such a query asks whether there are values for integer variables
  that make the constraint true.}.
The query encodes the model checking question:
whether a system---which is yet unknown---is accepted by the automaton.
Bounding the system size makes it possible to encode such a model checking query
into an SMT query.
To solve such a query,
an SMT solver efficiently enumerates every possible system of a given size,
and checks if it is correct.
Thus, if the SMT query is satisfiable, then we extract the system (of the given size),
otherwise increase the system size and repeat.
The loop stops when the bound on the system size%
---provided by the user or from the theory---is reached.

Our first bounded synthesiser for \CTLstar
resembles bottom-up \CTLstar model checking~\cite{PrinciplesMC}:
it introduces an atom for each subformula of the \CTLstar formula,
and encodes into an SMT query whether the atom holds in a system state,
for every state.
We also require the top-level atom, representing the whole \CTLstar formula,
to hold in the initial system state,
Hence, if the SMT query is satisfiable,
then there is a system of the given size,
which satisfies the \CTLstar formula.
Otherwise, increase the system size and repeat.

Our second bounded synthesiser for \CTLstar uses the automata framework~\cite{ATA}:
translate the \CTLstar formula into an alternating hesitant automaton,
then encode into an SMT query
whether there is a system of a given size that is accepted by the automaton.
Conceptually, the approach is the same as the previous one,
except that we do not introduce atoms for subformulas explicitly
and instead use their automata representation.

The results constitute Chapter~\ref{chap:bosy:ctlstar} and were published in:
\li
\-[\cite{CTLstarCAV}]
   \emph{Bounded Synthesis for Streett, Rabin, and \CTLstar},
   by Ayrat Khalimov and Roderick Bloem,
   at CAV conference, 2017
\il

\subsubsection*{Contribution I.2: \CTLstar-via-LTL Synthesis}

We reduce synthesis for \CTLstar properties to synthesis for LTL.
In the context of model checking this is impossible%
---\CTLstar is more expressive than LTL.
Yet, in synthesis we have knowledge of the system structure
\emph{and} we can add new outputs.
These outputs can be used to encode witnesses of
the satisfaction of \CTLstar subformulas directly into the system.
This way, we construct an LTL formula, over old and new outputs and original inputs,
which is realisable if, and only if, the original \CTLstar formula is realisable.
The \CTLstar-via-LTL synthesis approach preserves the problem complexity,
although it might produce systems that are larger than necessary.
Furthermore,
the approach directly benefits from the performance advances of LTL synthesisers.
The results constitute Chapter~\ref{chap:ctl-via-ltl} and were published in:
\li
\-[\cite{CTLsynt-via-LTLsynt}]
  \emph{\CTLstar Synthesis via LTL Synthesis},
  by Roderick Bloem and Sven Schewe and Ayrat Khalimov,
  at SYNT workshop, 2017
\il


\subsection*{Part II: Excursion into Parameterized Systems}

%\parbf{why do we consider parameterized systems?}

Modern systems become more and more distributed.
Distributed systems are hard to implement and even harder to debug.
Yet, the failure of such systems may be unacceptable.
Thus, substantial efforts are devoted to ensure the correctness of distributed systems.
In Part II,
we look into the hard task of automatic synthesis of distributed parameterized systems.

%\parbf{what is the parametrized synthesis problem?}

Most distributed systems, algorithms, and data structures are \emph{parameterized}:
they should work for a varied, not a priori fixed, number of the components.
%We call such systems parameterized (by the number of components).
The parameterized synthesis problem~\cite{JB14} asks, given
a parameterized specification, to find a process template,
that can be cloned to form a correctly behaving system of any size.
An example parameterized specification is:
\[ \begin{array}{ll}
  \forall i \neq j.~ & \G \neg ( g_i \land g_j ) \land \\
  \forall i.~ & \G (r_i \impl \F g_i).
  \end{array}
\]
The synthesizer should find a process template, having input $r$ and output $g$,
such that a system composed of any number of such processes,
satisfies the above specification.
The related question is that of parametrized model checking
where the process template is given.
The intrinsic parameter,
hidden in the parameterized synthesis problem,
is how the processes are connected and how they communicate,
i.e., the system architecture.
The survey of existing cutoff and decidability results
for many different system architectures can be found in~\cite{BloemETAL15}.
We focus on two system architectures: guarded systems and token-ring systems.

%\parbf{what is the standard solution to parameterized synthesis problem?}

A common approach to solve the parameterized synthesis and model checking problems
is to use the cutoff reduction~\cite{Emerso03}:
reduce reasoning about systems with an arbitrary number of processes
to reasoning about systems of a fixed cutoff size.
For example,
if we consider the parameterized specification mentioned above and token-ring systems,
then it is enough to consider a system with 4 processes:
if it is correct, then any larger system is correct.

%\parbf{what is our contribution?}

\subsubsection*{Contribution II.1: Cutoffs for Parameterized Guarded Systems}

Guarded systems~\cite{EmersonK03} are inspired by cache coherence protocols found
in most modern processors.
A cache coherence protocol is usually described by states,
where transitions between states happen depending on whether or not
there is a processor in a particular state.
I.e., the transitions are guarded.
Inspired by this, in guarded systems,
processes transitions are enabled or disabled depending
on the existence of other processes in certain local states.
Our contribution concerns both parameterized synthesis and parameterized verification.
Our work stems from the observation that
existing cutoff results for guarded systems
(i) are restricted to closed systems, and
(ii) are of limited use for liveness properties
because reductions do not preserve fairness.
We close these gaps and obtain new cutoff results for open systems with
liveness properties under fairness assumptions.
Furthermore, we obtain cutoffs for the detecting deadlocks,
which are of paramount importance in synthesis.
Finally, we prove tightness or asymptotic tightness for the new cutoffs.
The results constitute Chapter~\ref{chap:guarded-systems}
and were published in:
\li
\-[\cite{AJK16}]
   \emph{Tight Cutoffs for Guarded Protocols with Fairness},
   by Simon Au{\ss}erlechner and Swen Jacobs and Ayrat Khalimov,
   at VMCAI conference, 2016
\il

%\parbf{what is the problem with the existing parameterized synthesis?}

\subsubsection*{Contribution II.2: Case Study of Parameterized Token-ring AMBA}

In token-ring systems, a single token circulates in the system.
A process possessing the token knows that no other process has the token.
Based on this information, the process can, for example, raise the grant signal.
If all processes raise the grant only when they posses the token,
then the grants will be mutually exclusive.
Thus, the token serves as the resource token.

The experiments with the existing parameterized synthesis method~\cite{JB14}
showed that it does not scale to large specifications.
First, we optimize the method by refining the cutoff reduction.
The experiments show speed-ups of several orders of magnitude.
Second, we perform parameterized synthesis case study
on the industrial arbiter protocol AMBA~\cite{AMBAspec}.
We describe new cutoff extension and decompositional synthesis tailored to AMBA
that, together with the previously mentioned optimizations,
allowed us to synthesize AMBA in parameterized setting, for the first time.
The results constitute Chapter~\ref{chap:token-systems}
and were published in:
\li
\-[\cite{Khalimov13}]
   \emph{Towards Efficient Parameterized Synthesis},
   by Ayrat Khalimov and Swen Jacobs and Roderick Bloem,
   at VMCAI conference, 2013
\-[\cite{party}]
   \emph{PARTY: Parameterized Synthesis of Token Rings},
   by Ayrat Khalimov and Swen Jacobs and Roderick Bloem,
   at CAV conference, 2013
\-[\cite{BJK14}]
   \emph{Parameterized Synthesis Case Study: AMBA AHB},
   by Ayrat Khalimov and Swen Jacobs and Roderick Bloem,
   at SYNT workshop, 2014
\il

\subsection*{Other Results}

Here are the results that did not make their way into the thesis:
\li
\-[\cite{BloemETAL15}]
   \emph{Decidability of Parameterized Verification},
   book of 170 pages,
   by Roderick Bloem and
               Swen Jacobs and
               Ayrat Khalimov and
               Igor Konnov and
               Sasha Rubin and
               Helmut Veith and
               Josef Widder.\\
In this book we consider the important case of systems parameterized by the number of processes in the system
and where each process is independent of that number.
The literature in this area produced a wealth of computational models for systems based on token passing,
broadcast communication, guarded transitions, and other communication primitives.
We introduce a computational model that unites the central synchronization and
communication primitives of many models.
We survey existing decidability and undecidability results,
and provide a systematic overview of the basic problems in this research area.


\-[\cite{DBLP:journals/sigact/BloemJKKRVW16}]
   \emph{Decidability in Parameterized Verification},
   the journal version of the above book; appeared in SIGACT News in 2016.

\-[\cite{DBLP:journals/corr/Khalimov16}]
   \emph{Specification Format for Reactive Synthesis Problems},
   by Ayrat Khalimov,
   at SYNT workshop, 2015.\\
To do synthesis, we need a specification.
Writing specifications is hard.
In this paper, we propose a user-friendly format to ease
the specification work, in particularly, that of specifying partial implementations.
Also, we provide scripts to convert specifications in the new format into the SYNTCOMP format,
thus benefiting from state of the art synthesizers.

\-[\cite{AJKR14}]
   \emph{Parameterized Model Checking of Token-Passing Systems},
   by Benjamin Aminof and
               Swen Jacobs and
               Ayrat Khalimov and
               Sasha Rubin,
   at VMCAI conference, 2014.\\
In this paper, we revisit the parameterized model checking problem for token-passing
systems and specifications in indexed $\CTLstarmX$.
%In foundational work, Emerson and Namjoshi (1995, 2003) showed that
%parameterized model checking of indexed \CTLstarmX in uni-directional token rings
%can be reduced to checking rings up to some cutoff size.
%Clarke et al. (2004) showed a similar result for general topologies and indexed \LTLmX,
%provided processes cannot choose the directions for sending or receiving the token.
We unify and substantially extend the results of Emerson and Namjoshi~\cite{Emerso95b,Emerso03}
and Clarke et al.~\cite{Clarke04c}
by systematically exploring fragments of indexed \CTLstarmX with respect to general network topologies.
For each fragment we establish whether a cutoff exists,
and for some concrete topologies, such as rings, cliques and stars, we infer small cutoffs.
Finally, we show that the problem becomes undecidable, and thus no cutoffs exist,
if processes are allowed to choose the directions in which they send or from which they receive the token.

\-[\cite{2017arXiv171204291K}]
  \emph{OpenSEA: Semi-Formal Methods for Soft Error Analysis},
  by Patrick Klampfl and Robert K\"onighofer and Roderick Bloem and Ayrat Khalimov and
  Aiman Abu-Yonis  and Shiri Moran,
  on arxiv, 2017.\\
Due to alpha-particles and cosmic rays, modern circuits are prone to bit flips.
To alleviate the problem, designers develop protection circuits,
but they are hard to implement right.
This leads to bugs: an undetected fault can bring miscalculations,
the protection that alarms about harmless faults incurs performance penalty.
In this paper, we use formal methods on designer’s input tests, while keeping time-location open.
This idea is at the core of the tool OpenSEA.
OpenSEA can
(i) find latches vulnerable to and protected against faults,
(ii) find tests that exhibit checker false alarms,
(iii) use fixed and open inputs, and
(iv) use environment assumptions.
Evaluation on a number of industrial designs shows that OpenSEA produces valuable results.
\il
