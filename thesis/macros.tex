\makeatletter
\if@titlepage
  \newenvironment{abstract}{%
      \titlepage
      \null\vfil
      \@beginparpenalty\@lowpenalty
      \begin{center}%
        \bfseries \abstractname
        \@endparpenalty\@M
      \end{center}}%
     {\par\vfil\null\endtitlepage}
\else
  \newenvironment{abstract}{%
      \if@twocolumn
        \section*{\abstractname}%
      \else
        \small
        \begin{center}%
          {\bfseries \abstractname\vspace{-.5em}\vspace{\z@}}%
        \end{center}%
        \quotation
      \fi}
      {\if@twocolumn\else\endquotation\fi}
\fi
\makeatother



\usepackage{acronym}
\usepackage{booktabs}
\usepackage{pdflscape}
\usepackage{epigraph}

\let\proof\relax
\let\endproof\relax

\usepackage{amsthm}
\usepackage{amsfonts}
\usepackage{amsmath}
\usepackage{amssymb}
\usepackage{temporal}
\usepackage{graphicx}
\usepackage{caption}
\usepackage{xspace}
\usepackage{subcaption}
\usepackage{stmaryrd} % for short arrows
%\usepackage{parskip}  % don't indent paragraphs -- use vertical spacing instead
\usepackage[svgnames]{xcolor}  % used by preamble
\usepackage{multirow}
\usepackage{doi}   % to allow _ in doins
\usepackage{tikz}\usetikzlibrary{arrows,automata,calc,fit,positioning}
\usepackage{wrapfig}
\usepackage{booktabs}  % better tables
\usepackage{colortbl}
\usepackage{float}  % for forcing position with 'H' (like \begin{table}[H])
\usepackage{environ}  % Guarded Systems use \NewEnv to create new env for tikz pics
\usepackage{enumitem}  % Guarded Systems
\usepackage{mdframed}
\usepackage{wasysym}
\usepackage{pifont}

\usepackage{thmtools, thm-restate}

% for appendix at the end of the chapter
% + some tricks for TOC
% https://tex.stackexchange.com/questions/120716/appendix-after-each-chapter
\usepackage{appendix}
\usepackage{chngcntr}
\usepackage{etoolbox}
\AtBeginEnvironment{subappendices}{%
\chapter*{Appendix}
\addcontentsline{toc}{chapter}{~~~~Appendix}
\counterwithin{figure}{section}
\counterwithin{table}{section}
}

%%%%%%%%%%%%%% trackchanges %%%%%%%%%%%%%%
% ignore llncs note command
\let\note\relax
\let\endnote\relax

\iffinal
  \usepackage[finalold]{trackchanges} % also use \sr,\sj,\ak,\ba
\else
  \usepackage[inline]{trackchanges}
\fi

\addeditor{$A\!K$}
\addeditor{$R\!B$}
\addeditor{$S\!J$}
\newcommand{\ak}[1]{\note[$A\!K$]{#1}}
\newcommand{\rb}[1]{\note[$R\!B$]{#1}}
\newcommand{\sj}[1]{\note[$S\!J$]{#1}}
%% end of trackchanges

\usepackage{hyperref}   % goes last
\hypersetup{
    colorlinks,
    linkcolor=black,
    citecolor=black,
}

\makeatletter
\newcommand\footnoteref[1]{\protected@xdef\@thefnmark{\ref{#1}}\@footnotemark}
\makeatother

%---------------------------------------------------------------------
%---------------------------------------------------------------------
%---------------------------------------------------------------------
%---------------------------------------------------------------------
%---------------------------------------------------------------------
%---------------------------------------------------------------------


\newcommand{\mailto}[1]{\href{mailto:#1}{\nolinkurl{#1}}}
\newcommand{\ack}[1]{\smallskip\noindent\small{\textbf{Acknowledgements.} #1}}


\newcommand\abstractname{Abstract}
%\newtheorem{obs}{Observation}
%\declaretheorem[name=Theorem]{thm}
%\declaretheorem[name=Lemma]{lem}
%\declaretheorem[name=Corollary]{cor}
%\declaretheorem[name=Observation]{obs}
%\declaretheorem[name=Claim]{claim}

% BEGIN theorems
\newtheoremstyle{myexample}% name
  {5pt}%      Space above
  {5pt}%      Space below
  {}%         Body font
  {}%         Indent amount (empty = no indent, \parindent = para indent)
  {\itshape}% Thm head font
  {.}%        Punctuation after thm head
  {.5em}%     Space after thm head: " " = normal interword space;
        %       \newline = linebreak
  {}%         Thm head spec (can be left empty, meaning `normal')
{\theoremstyle{myexample} \newtheorem{example}{Example} }
\newtheorem{theorem}{Theorem}
\newtheorem{lemma}[theorem]{Lemma}
\newtheorem{proposition}[theorem]{Proposition}
\newtheorem{corollary}[theorem]{Corollary}
{\theoremstyle{definition} \newtheorem{remark}{Remark} }
%\newenvironment{proof}{\begin{IEEEproof}}{\end{IEEEproof}}
%\newcommand{\qed}{}



%paragraph with some skip and bold heading
\newcommand{\parbf}[1]{\smallskip\noindent {\bf #1.}}
\newcommand{\parit}[1]{\smallskip\noindent {\it #1.}}

\newcommand{\bbN}{\mathbb{N}}
\newcommand{\bbB}{\mathbb{B}}
\newcommand{\bbR}{\mathbb{R}}
\newcommand{\bigdotcup}{\ensuremath{\mathaccent\cdot{\bigcup}}}

\newcommand\LTL{\ensuremath{\text{LTL}}\xspace}
\newcommand\LTLmX{\ensuremath{\text{LTL}\backslash\text{X}}}
\newcommand\CTLstar{\ensuremath{{\text{CTL}^*}}\xspace}
\newcommand\CTLstarmX{\ensuremath{{\text{CTL}^*\backslash\text{X}}}\xspace}
\newcommand\CTL{\ensuremath{\text{CTL}}\xspace}

\definecolor{lightgray}{gray}{0.8}
\definecolor{darkgreen}{rgb}{0,0.5,0}
\definecolor{darkblue}{rgb}{0,0,.5}
\definecolor{darkred}{rgb}{0.9,0,0}
\definecolor{pink}{rgb}{0.95,0.08,0.55}
\definecolor{mygray}{gray}{.6}
\newcommand{\minor}[1]{{\footnotesize\textcolor{mygray}{#1}}}
%% end of trackchanges


%% 'such that'
\renewcommand{\|}{\mid}
%% disjoint union
\newcommand{\cupdot}{\mathbin{\dot{\cup}}}


%% specification, implication
\renewcommand{\iff}{\leftrightarrow}
\newcommand{\Iff}{\Leftrightarrow}
\newcommand{\impl}{\rightarrow}
\newcommand{\Impl}{\Rightarrow}
\newcommand{\Implied}{\Leftarrow}

% add line breaks to table cells, to labels of tikz 
% usage: \specialcellC{first line \\ second line} 
\newcommand{\specialcellC}[2][c]{%
  \begin{tabular}[#1]{@{}c@{}}#2\end{tabular}}
\newcommand{\specialcellL}[2][c]{%
  \begin{tabular}[#1]{@{}l@{}}#2\end{tabular}}

% temporal logic
\newcommand\G\always\xspace
\newcommand\F\eventually\xspace
\newcommand\W\weakuntil\xspace
\newcommand\U\until\xspace
\newcommand\R\releases\xspace
\newcommand\X\nextt\xspace
\newcommand\XX{\ensuremath{\nextt\!\nextt\xspace}}
\newcommand\XXX{\ensuremath{\nextt\!\nextt\!\nextt\xspace}}
\newcommand{\GF}{\G\!\F\xspace}
\newcommand{\FG}{\F\!\G\xspace}
\newcommand\E{{\ensuremath\pexists}\xspace}
\newcommand\A{{\ensuremath\pforall}\xspace}


% Latex syntax sugar
\newcommand\li{\begin{itemize}}
\newcommand\il{\end{itemize}}
\renewcommand{\-}{\item}
%\newcommand\lo{\begin{enumerate}}
%\newcommand\ol{\end{enumerate}}

\newcommand\todo[1]{\ak{\hl{todo: #1}}}

% ttfamily with bold. Use in lstslistings: 
% \begin{lstlisting}[basicstyle=\ttfamilywithbold,language=python,mathescape]
%   the code here will have bold
% \end{lstlisting}
% http://tex.stackexchange.com/questions/25249/how-do-i-use-a-particular-font-for-a-small-section-of-text-in-my-document/25251#25251
%\newcommand*{\ttfamilywithbold}{\fontfamily{lmtt}\selectfont}

\newcommand{\current}{\sethlcolor{green}\ \hl{current} \ \sethlcolor{yellow}}

\newcommand{\tpl}[1]{\left<#1\right>}

\newcommand{\edge}[3]{#1 \stackrel{{#2}}{\rightarrow} #3}
\newcommand{\trans}[1]{\stackrel{{#1}}{\rightarrow}}
\newcommand{\eqbydef}{\triangleq}

\newcommand{\I}{{2^I}}
\renewcommand{\O}{{2^O}}
\newcommand{\reach}{\mathit{rch}}
\newcommand{\rank}{\rho}
\newcommand{\RANK}{\ensuremath{R_\triangleright}\xspace}

% types of properties
\newcommand{\buchi}{B\"{u}chi\xspace}
\newcommand{\AUBW}{\ensuremath{\A(\textit{UBW})}\xspace}
\newcommand{\AUCW}{\ensuremath{\A(\textit{UCW})}\xspace}
\newcommand{\AUSW}{\ensuremath{\A(\textit{USW})}\xspace}
\newcommand{\AURW}{\ensuremath{\A(\textit{URW})}\xspace}
\newcommand{\AUXW}{\ensuremath{\A(\textit{UXW})}\xspace}

\newcommand{\ENBW}{\ensuremath{\E(\textit{NBW})}\xspace}
\newcommand{\NBW}{\ensuremath{\textit{NBW}}\xspace}
\newcommand{\AHT}{\ensuremath{\textit{AHT}}\xspace}
\newcommand{\ENCW}{\ensuremath{\E(\textit{NCW})}\xspace}
\newcommand{\ENSW}{\ensuremath{\E(\textit{NSW})}\xspace}
\newcommand{\EDSW}{\ensuremath{\E(\textit{DSW})}\xspace}
\newcommand{\ADSW}{\ensuremath{\A(\textit{DSW})}\xspace}
\newcommand{\EDRW}{\ensuremath{\E(\textit{DRW})}\xspace}
\newcommand{\EDFW}{\ensuremath{\E(\textit{DFW})}\xspace}
\newcommand{\EDBW}{\ensuremath{\E(\textit{DBW})}\xspace}
\newcommand{\ENRW}{\ensuremath{\E(\textit{NRW})}\xspace}
\newcommand{\ENXW}{\ensuremath{\E(\textit{NXW})}\xspace}

% other math macros
\newcommand{\out}{\mathit{out}}
\newcommand{\GR}{\triangleright}
\newcommand{\PhiE}{\Phi^{\GR}_\E}
\newcommand{\PhiA}{\Phi^{\GR}_\A}

\newcommand{\Inf}{\textit{Inf}}
\newcommand{\Fin}{\textit{Fin}}
\newcommand{\Pref}{\textit{Pref}}

\renewcommand{\true}{\textsf{true}\xspace}
\renewcommand{\false}{\textsf{false}\xspace}


% medium sized versions of operators
\DeclareMathOperator*{\medoplus}{\text{\raisebox{0.25ex}{\scalebox{0.8}{$\bigoplus$}}}}
\DeclareMathOperator*{\medvee}{\text{\raisebox{0.25ex}{\scalebox{0.8}{$\bigvee$}}}}
\DeclareMathOperator*{\medwedge}{\text{\raisebox{0.25ex}{\scalebox{0.8}{$\bigwedge$}}}}


%%%%%%%%%%%%%%%%%%%%%%%%%%%%%%%%%%%%%%%%%%%%%%%%%%%%%%%%%%%%
% preamble used for pictures by tikzit
% I don't know how to import the preamble though:
% of course, you can just copy-paste it into the custom preamble,
% but the default preamble should also be changed so that tikzit can correctly display it
%    but tikzit does not allow to modify the default preamble nor to import it...

%\usepackage[svgnames]{xcolor}
\usepackage{tikz}
\usetikzlibrary{arrows,automata}
\usetikzlibrary{decorations.markings}
\usetikzlibrary{shapes.geometric}

% for some reason,
% in this book class all fonts are larger than in the non-book version,
% so let's set font=\small
\tikzset{every picture/.style={/utils/exec={\small}}}  

\pgfdeclarelayer{edgelayer}
\pgfdeclarelayer{nodelayer}
\pgfsetlayers{edgelayer,nodelayer,main}

\tikzstyle{none}=[inner sep=0pt]
\tikzset{initial text={}}

\tikzstyle{rn}=[circle,fill=White,draw=Red,minimum size=0.4cm,inner sep=0pt]
\tikzstyle{gn}=[circle,fill=White,draw=Green,minimum size=0.4cm,inner sep=0pt]
\tikzstyle{yn}=[circle,fill=White,draw=Yellow,minimum size=0.4cm,inner sep=0pt]
\tikzstyle{wn}=[circle,fill=White,draw=Black,minimum size=0.4cm,inner sep=0pt]
\tikzstyle{invisible}=[circle,fill=White,draw=White,inner sep=0pt]
\tikzstyle{textual}=[rectangle]
\tikzstyle{dot}=[circle,fill=Black,draw=Black,inner sep=0pt,minimum size=0.5mm]
\tikzstyle{uptriangle}=[regular polygon,regular polygon sides=3,shape border rotate=0,fill=Black,draw=Black,inner sep=0pt,minimum size=1mm]
\tikzstyle{text ellipse}=[rectangle,rounded corners,fill=White,draw=Black,minimum size=0.6cm,inner sep=0pt]



\tikzstyle{simple}=[-,draw=Black]
\tikzstyle{arrow}=[->,draw=Black]
\tikzstyle{dashed arrow}=[->,draw=Black,dashed]
\tikzstyle{tick}=[-,draw=Black,postaction={decorate},decoration={markings,mark=at position .5 with {\draw (0,-0.1) -- (0,0.1);}}]
\tikzstyle{ga}=[->,draw=Green]
\tikzstyle{pa}=[->,draw=DeepPink]
\tikzstyle{red}=[-,draw=Red]
\tikzstyle{pink}=[-,draw=DeepPink]
\tikzstyle{blue}=[-,draw=Blue]
\tikzstyle{green}=[-,draw=Green]
\tikzstyle{yellow}=[-,draw=DarkGoldenrod]










%%%% end of tikzit preamble
%%%%%%%%%%%%%%%%%%%%%%%%%%%%%%%%%%%%%%%%%%%%%%%%%%%%%%%%%%%%
